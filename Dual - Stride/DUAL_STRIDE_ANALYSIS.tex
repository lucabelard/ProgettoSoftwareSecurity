\documentclass[12pt,a4paper]{article}

% Pacchetti essenziali
\usepackage[utf8]{inputenc}
\usepackage[italian]{babel}
\usepackage[T1]{fontenc}
\usepackage{graphicx}
\usepackage{amsmath}
\usepackage{amssymb}
\usepackage{hyperref}
\usepackage{listings}
\usepackage{xcolor}
\usepackage{geometry}
\usepackage{fancyhdr}
\usepackage{booktabs}
\usepackage{float}
\usepackage{caption}
\usepackage{subcaption}

% Configurazione geometria pagina
\geometry{
    a4paper,
    left=3cm,
    right=3cm,
    top=3cm,
    bottom=3cm
}

% Configurazione header e footer
\pagestyle{fancy}
\fancyhf{}
\fancyhead[L]{\leftmark}
\fancyhead[R]{\thepage}
\renewcommand{\headrulewidth}{0.4pt}

% Configurazione listing per codice Solidity
\lstdefinelanguage{Solidity}{
    keywords={contract, function, public, private, external, internal, view, pure, payable, require, emit, event, struct, mapping, uint256, uint8, address, bool, bytes32, returns, if, else, return, storage, memory, calldata},
    keywordstyle=\color{blue}\bfseries,
    ndkeywords={class, export, boolean, throw, implements, import, this},
    ndkeywordstyle=\color{darkgray}\bfseries,
    identifierstyle=\color{black},
    sensitive=false,
    comment=[l]{//},
    morecomment=[s]{/*}{*/},
    commentstyle=\color{purple}\ttfamily,
    stringstyle=\color{red}\ttfamily,
    morestring=[b]',
    morestring=[b]"
}

\lstset{
    language=Solidity,
    basicstyle=\small\ttfamily,
    numbers=left,
    numberstyle=\tiny,
    stepnumber=1,
    numbersep=5pt,
    backgroundcolor=\color{lightgray!10},
    showspaces=false,
    showstringspaces=false,
    showtabs=false,
    frame=single,
    tabsize=2,
    captionpos=b,
    breaklines=true,
    breakatwhitespace=false,
    escapeinside={\%*}{*)},
    xleftmargin=2em,
    framexleftmargin=1.5em
}

% Hyperref setup
\hypersetup{
    colorlinks=true,
    linkcolor=blue,
    filecolor=magenta,      
    urlcolor=cyan,
    citecolor=green,
    pdftitle={Analisi di Sicurezza DUAL-STRIDE-DUA},
    pdfauthor={Luca Belard},
}

% Comandi personalizzati
\newcommand{\code}[1]{\texttt{#1}}
\newcommand{\stride}{STRIDE}
\newcommand{\dua}{DUA}

\begin{document}

% ===== FRONTESPIZIO =====
\begin{titlepage}
    \centering
    \vspace*{2cm}
    
    {\LARGE\bfseries Università degli Studi\par}
    \vspace{0.5cm}
    {\Large Dipartimento di Informatica\par}
    \vspace{2cm}
    
    {\Huge\bfseries Analisi di Sicurezza DUAL-STRIDE-DUA\par}
    \vspace{0.5cm}
    {\Large Sistema Oracolo Bayesiano per la Catena del Freddo Farmaceutica\par}
    \vspace{2cm}
    
    {\large
    \textbf{Relazione Tecnica}\\
    Versione 3.0 -- Modular \& Enhanced
    \par}
    
    \vfill
    
    {\large
    \textbf{Autore:}\\
    Luca Belard
    \par}
    
    \vspace{1cm}
    
    {\large 27 Gennaio 2026\par}
\end{titlepage}

% ===== ABSTRACT =====
\newpage
\begin{abstract}
La presente relazione descrive un'analisi approfondita di sicurezza applicata a un sistema blockchain innovativo per la gestione della catena del freddo farmaceutica. Il sistema, basato su architettura Ethereum e reti bayesiane, implementa un meccanismo di pagamento automatico condizionato alla validazione di evidenze provenienti da sensori IoT distribuiti lungo la catena logistica.

L'analisi di sicurezza è stata condotta seguendo la metodologia \stride-\dua{} (Spoofing, Tampering, Repudiation, Information Disclosure, Denial of Service, Elevation of Privilege, Danger, Unreliability, Absence of resilience), un framework particolarmente adatto per sistemi cyber-fisici critici. Questa metodologia è stata applicata con un approccio duale, considerando separatamente le minacce da attaccanti intenzionali e quelle derivanti da utenti maldestri.

Il documento presenta inoltre le contromisure implementate nella versione 3.0 del sistema, che includono un'architettura modulare, meccanismi di offuscamento dei dati sensibili, un sistema completo di rimborso con protezione anti-DoS, e un framework di runtime monitoring per la rilevazione di anomalie e violazioni di sicurezza.

I risultati mostrano un significativo miglioramento del profilo di sicurezza del sistema: le vulnerabilità critiche sono state ridotte da tre a zero, mentre le vulnerabilità di livello alto sono diminuite del 60\%. L'implementazione delle contromisure proposte ha permesso di risolvere completamente il problema critico del blocco indefinito dei fondi in escrow e di proteggere efficacemente i dati commerciali sensibili attraverso tecniche di hashing on-chain.
\end{abstract}

% ===== INDICE =====
\newpage
\tableofcontents

% ===== INTRODUZIONE =====
\newpage
\section{Introduzione}

\subsection{Contesto e Motivazioni}

La gestione della catena del freddo nel settore farmaceutico rappresenta una sfida critica per la sicurezza dei pazienti e l'efficacia terapeutica dei prodotti farmaceutici. Durante il trasporto, i farmaci termolabili devono mantenere condizioni di temperatura rigidamente controllate, tipicamente nell'intervallo 2-8°C. Il mancato rispetto di queste condizioni può comprometterne irreversibilmente l'efficacia, con conseguenze potenzialmente gravi per la salute pubblica.

I sistemi tradizionali di monitoraggio della catena del freddo si basano su processi cartacei o database centralizzati, soggetti a diverse criticità: possibilità di manomissione dei dati, mancanza di trasparenza, inefficienza nei processi di verifica e dispute frequenti tra le parti coinvolte. Inoltre, il meccanismo di pagamento tradizionale non prevede alcuna garanzia automatica legata alla qualità del servizio erogato, creando asimmetrie informative e conflitti di interesse.

Per affrontare queste problematiche, è stato sviluppato un sistema innovativo basato su tecnologia blockchain che integra smart contract Ethereum con sensori IoT e reti bayesiane per la validazione probabilistica della conformità. Questo approccio offre vantaggi significativi in termini di immutabilità dei dati, trasparenza, automazione dei pagamenti condizionati e riduzione delle dispute.

\subsection{Scopo della Relazione}

La presente relazione ha come obiettivo principale l'analisi approfondita della sicurezza del sistema proposto, con particolare attenzione all'identificazione sistematica delle minacce e alla validazione delle contromisure implementate. L'analisi è stata condotta utilizzando il framework \stride-\dua, una metodologia consolidata per l'analisi delle minacce nei sistemi informatici, estesa con componenti specifici per sistemi cyber-fisici critici.

L'approccio adottato prevede:
\begin{itemize}
    \item L'identificazione e catalogazione di tutti gli asset critici del sistema
    \item La definizione di un modello delle minacce che considera sia attaccanti sofisticati che utenti maldestri
    \item L'applicazione sistematica delle categorie \stride-\dua{} a ciascun asset identificato
    \item La mappatura delle minacce ai pattern di attacco noti (CAPEC e MITRE ATT\&CK)
    \item La documentazione delle contromisure implementate e la valutazione della loro efficacia
    \item La proposta di raccomandazioni per il miglioramento continuo della posture di sicurezza
\end{itemize}

\subsection{Organizzazione del Documento}

Il documento è organizzato come segue. Il Capitolo 2 presenta la metodologia di analisi adottata, descrivendo in dettaglio il framework \stride-\dua{} e le sue estensioni. Il Capitolo 3 fornisce una descrizione tecnica dell'architettura del sistema, con particolare enfasi sugli asset critici e sull'architettura modulare implementata nella versione 3.0. Il Capitolo 4 definisce il modello delle minacce, identificando gli attori potenzialmente ostili e le loro capacità. Il Capitolo 5 costituisce il nucleo dell'analisi, presentando l'applicazione sistematica delle categorie \stride-\dua{} a ciascun asset, con riferimenti ai pattern di attacco documentati nella letteratura. Il Capitolo 6 documenta le contromisure implementate nel sistema, mentre il Capitolo 7 propone raccomandazioni per implementazioni future. Infine, il Capitolo 8 presenta le conclusioni e una valutazione complessiva dello stato di sicurezza del sistema.

\subsection{Contributi Principali}

I principali contributi di questa relazione possono essere sintetizzati nei seguenti punti:

\begin{enumerate}
    \item \textbf{Analisi sistematica della sicurezza}: applicazione rigorosa della metodologia \stride-\dua{} a un sistema blockchain per la supply chain farmaceutica, con particolare attenzione agli aspetti cyber-fisici.
    
    \item \textbf{Architettura modulare di sicurezza}: progettazione e implementazione di un'architettura a tre livelli (BNCore, BNGestoreSpedizioni, BNPagamenti) che realizza il principio di separazione delle responsabilità e riduce la superficie di attacco.
    
    \item \textbf{Meccanismi di protezione dei fondi}: implementazione di un sistema completo di rimborso con tre meccanismi distinti (annullamento precoce, timeout automatico, protezione anti-frode) che elimina completamente il rischio di blocco indefinito dei fondi in escrow.
    
    \item \textbf{Privacy-by-design}: integrazione di tecniche di offuscamento dei dati sensibili attraverso hashing on-chain, bilanciando le esigenze di trasparenza blockchain con la protezione dei segreti commerciali.
    
    \item \textbf{Runtime monitoring}: implementazione di un framework di eventi per il monitoraggio continuo di violazioni di sicurezza e anomalie operative.
    
    \item \textbf{Validazione empirica}: dimostrazione quantitativa del miglioramento della posture di sicurezza, con riduzione da 3 a 0 delle vulnerabilità critiche e diminuzione del 60\% delle vulnerabilità di livello alto.
\end{enumerate}

% ===== METODOLOGIA =====
\newpage
\section{Metodologia di Analisi}

\subsection{Il Framework STRIDE}

STRIDE è un framework di threat modeling sviluppato da Microsoft e ampiamente adottato nell'industria del software per l'identificazione sistematica delle minacce alla sicurezza. L'acronimo STRIDE identifica sei categorie di minacce fondamentali:

\begin{description}
    \item[Spoofing (Falsificazione dell'identità)] Minacce in cui un attaccante si spaccia per un'entità legittima del sistema. Nel contesto blockchain, questo può includere l'impersonificazione di ruoli attraverso il furto o la compromissione delle chiavi private.
    
    \item[Tampering (Manomissione)] Modifiche non autorizzate a dati o codice del sistema. Nonostante l'immutabilità della blockchain, i dati off-chain e i parametri configurabili rimangono vulnerabili a questa categoria di attacchi.
    
    \item[Repudiation (Ripudio)] Capacità di un attore di negare di aver compiuto un'azione. La blockchain offre naturalmente protezione contro questa minaccia attraverso la tracciabilità crittografica delle transazioni.
    
    \item[Information Disclosure (Divulgazione di informazioni)] Esposizione non autorizzata di informazioni sensibili. La natura pubblica della blockchain Ethereum rende questa categoria particolarmente rilevante per i dati commerciali.
    
    \item[Denial of Service (Negazione del servizio)] Attacchi che rendono il sistema non disponibile o non funzionale. Nel contesto degli smart contract, questo include anche il blocco di fondi attraverso logica difettosa.
    
    \item[Elevation of Privilege (Escalation dei privilegi)] Ottenimento non autorizzato di permessi elevati. Nei sistemi basati su AccessControl, questo rappresenta una minaccia critica.
\end{description}

\subsection{Estensione DUA per Sistemi Cyber-Fisici}

Il framework STRIDE classico è stato originariamente concepito per sistemi puramente software. Per sistemi cyber-fisici critici, è necessario considerare categorie aggiuntive di minacce. L'estensione DUA introduce tre nuove dimensioni di analisi:

\begin{description}
    \item[Danger (Pericolo fisico)] Minacce che possono causare danni fisici a persone o proprietà. Nel contesto della catena del freddo farmaceutica, questo include scenari in cui prodotti deteriorati per mancato controllo della temperatura vengono comunque validati e distribuiti.
    
    \item[Unreliability (Inaffidabilità)] Comportamenti imprevedibili o erratici di componenti del sistema. Particolarmente rilevante per i sensori IoT, che possono guastarsi, fornire letture errate o perdere la connettività.
    
    \item[Absence of Resilience (Assenza di resilienza)] Incapacità del sistema di mantenere funzionalità critiche in presenza di guasti o attacchi. Questa categoria valuta la robustezza complessiva dell'architettura.
\end{description}

\subsection{Approccio Duale: Attaccanti vs. Utenti Maldestri}

Una caratteristica distintiva di questa analisi è l'adozione di un approccio duale che distingue tra:

\begin{itemize}
    \item \textbf{Attaccanti intenzionali}: Soggetti con competenze tecniche e motivazione economica (o di altro tipo) che cercano deliberatamente di compromettere il sistema. Questo include hacker esterni, insider malevoli e organizzazioni concorrenti.
    
    \item \textbf{Utenti maldestri}: Utilizzatori legittimi del sistema che commettono errori non intenzionali a causa di inesperienza, incomprensione del sistema, o semplice disattenzione. Questi scenari sono particolarmente rilevanti per sistemi blockchain, dove errori operativi (come la perdita di chiavi private) possono avere conseguenze irreversibili.
\end{itemize}

Questa distinzione permette di progettare contromisure appropriate per entrambe le categorie: controlli tecnici di sicurezza per gli attaccanti intenzionali, e meccanismi di usabilità e prevenzione degli errori per gli utenti maldestri.

\subsection{Integrazione con Pattern di Attacco Documentati}

Per ogni minaccia identificata attraverso l'applicazione del framework \stride-\dua, l'analisi include riferimenti a:

\begin{itemize}
    \item \textbf{CAPEC (Common Attack Pattern Enumeration and Classification)}: Un catalogo pubblico di pattern di attacco comuni, mantenuto da MITRE Corporation, che descrive le metodologie utilizzate dagli attaccanti.
    
    \item \textbf{MITRE ATT\&CK}: Un framework di tattiche e tecniche basato su osservazioni reali di comportamenti avversari. Originariamente sviluppato per sistemi enterprise, fornisce un linguaggio comune per descrivere le azioni degli attaccanti.
\end{itemize}

Questa mappatura permette di contestualizzare le minacce identificate all'interno del più ampio panorama della cybersecurity e di beneficiare delle conoscenze e delle best practice consolidate nella comunità di sicurezza informatica.

% Aggiungiamo la sezione Architettura del Sistema
\newpage
\section{Architettura del Sistema}

\subsection{Panoramica Generale}

Il sistema oggetto di questa analisi implementa un meccanismo di pagamento condizionato (escrow) sulla blockchain Ethereum, integrato con sensori IoT per il monitoraggio in tempo reale delle condizioni di trasporto farmaceutico. L'architettura si basa su tre componenti principali:

\begin{enumerate}
    \item \textbf{Smart Contract}: Implementato in Solidity, gestisce la logica di business, i pagamenti in Ether, e il calcolo dell'inferenza bayesiana
    \item \textbf{Rete di Sensori IoT}: Dispositivi hardware che monitorano temperatura, integrità dei sigilli, shock fisici, esposizione alla luce e arrivo a destinazione
    \item \textbf{Interfaccia Web}: Applicazione front-end che permette l'interazione degli utenti con lo smart contract
\end{enumerate}

Il flusso operativo prevede che un mittente (farmacia) crei una spedizione depositando un importo in Ether come pagamento per il corriere. Durante il trasporto, i sensori IoT inviano periodicamente evidenze sullo stato della spedizione. Al termine del trasporto, un algoritmo bayesiano valuta la probabilità che le condizioni di trasporto siano state rispettate. Se la probabilità supera una soglia predefinita del 95\%, il pagamento viene automaticamente trasferito al corriere; altrimenti, il mittente può richiedere un rimborso.

\subsection{Evoluzione Architetturale: dalla Monolite al Modulare}

Nella versione iniziale (v1.0 e v2.0), lo smart contract era implementato come un unico contratto monolitico che gestiva tutte le funzionalità: calcoli bayesiani, gestione spedizioni, invio evidenze e pagamenti. Questa architettura presentava diverse criticità dal punto di vista della sicurezza e della manutenibilità:

\begin{itemize}
    \item Elevata complessità del codice, con conseguente difficoltà di audit
    \item Superficie di attacco estesa
    \item Difficoltà nel testing di componenti individuali
    \item Impossibilità di upgrade selettivo di singole funzionalità
\end{itemize}

La versione 3.0 introduce un'architettura modulare basata su ereditarietà gerarchica di contratti Solidity, seguendo il principio di separazione delle responsabilità (Separation of Concerns). L'architettura è strutturata su tre livelli:

\subsubsection{Livello 1: BNCore -- Logica Bayesiana}

Il contratto \code{BNCore} costituisce la base dell'architettura e implementa esclusivamente la logica di inferenza bayesiana. Le responsabilità di questo modulo includono:

\begin{itemize}
    \item Gestione delle Conditional Probability Tables (CPT) per ciascuna evidenza
    \item Implementazione dell'algoritmo di calcolo delle probabilità a posteriori
    \item Definizione dei ruoli di base e delle costanti del sistema (soglia di probabilità, precisione)
    \item Eventi per il logging delle modifiche ai parametri bayesiani
\end{itemize}

Una caratteristica fondamentale di sicurezza implementata a questo livello è la ren dizione privata delle CPT. Nella v2.0, le tabelle di probabilità condizionata erano pubblicamente leggibili dalla blockchain, permettendo ad attaccanti di effettuare reverse-engineering dei requisiti di conformità e simulare offline quali combinazioni di evidenze fossero sufficienti per ottenere un pagamento. Nella v3.0, le CPT sono dichiarate \code{private} e accessibili solo tramite getter protetti dal modificatore \code{onlyRole(DEFAULT\_ADMIN\_ROLE)}.

Il listato \ref{lst:bncore} mostra la struttura essenziale del contratto BNCore.

\begin{lstlisting}[caption={Struttura del contratto BNCore},label={lst:bncore}]
contract BNCore is AccessControl {
    // Costanti
    uint256 public constant PRECISIONE = 100;
    uint8 public constant SOGLIA_PROBABILITA = 95;
    
    // Ruoli
    bytes32 public constant RUOLO_ORACOLO = keccak256("RUOLO_ORACOLO");
    
    // Probabilita a priori
    uint256 public p_F1_T; // P(F1=True)
    uint256 public p_F2_T; // P(F2=True)
    
    // CPT private - protezione reverse-engineering
    CPT private cpt_E1;
    CPT private cpt_E2;
    CPT private cpt_E3;
    CPT private cpt_E4;
    CPT private cpt_E5;
    
    // Getter protetti (solo admin)
    function getCPT_E1() external view 
        onlyRole(DEFAULT_ADMIN_ROLE) 
        returns (CPT memory) 
    {
        return cpt_E1;
    }
    
    // Calcolo probabilita posteriori (protetto, internal)
    function _calcolaProbabilitaPosteriori(StatoEvidenze memory evidenze) 
        internal view 
        returns (uint256, uint256) 
    {
        // Implementazione inferenza bayesiana
        // ...
    }
}
\end{lstlisting}

