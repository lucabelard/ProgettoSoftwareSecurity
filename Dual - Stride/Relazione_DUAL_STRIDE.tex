\documentclass[12pt,a4paper]{article}

% Pacchetti essenziali
\usepackage[utf8]{inputenc}
\usepackage[italian]{babel}
\usepackage[T1]{fontenc}
\usepackage{graphicx}
\usepackage{amsmath}
\usepackage{amssymb}
\usepackage{hyperref}
\usepackage{listings}
\usepackage[table]{xcolor}
\usepackage{colortbl}
\usepackage{geometry}
\usepackage{fancyhdr}
\usepackage{booktabs}
\usepackage{float}
\usepackage{caption}
\usepackage{subcaption}
\usepackage{pdflscape}
\usepackage{enumitem}
\usepackage{longtable}
\usepackage{multirow}

% Colori personalizzati per le tabelle
\definecolor{headerblue}{RGB}{51, 102, 204} % Blu scuro
\definecolor{labelblue}{RGB}{204, 229, 255}  % Blu chiaro

% Configurazione geometria pagina
\geometry{
    a4paper,
    left=2.5cm,
    right=2.5cm,
    top=2.5cm,
    bottom=2.5cm
}

% Configurazione header e footer
\pagestyle{fancy}
\fancyhf{}
\fancyhead[L]{\leftmark}
\fancyhead[R]{\thepage}
\renewcommand{\headrulewidth}{0.4pt}

% Code Listing Settings
\lstdefinelanguage{Solidity}{
    keywords={contract, function, public, private, external, internal, view, pure, payable, require, emit, event, struct, mapping, uint256, uint8, address, bool, bytes32, returns, if, else, return, storage, memory, calldata},
    keywordstyle=\color{blue}\bfseries,
    identifierstyle=\color{black},
    sensitive=false,
    comment=[l]{//},
    morecomment=[s]{/*}{*/},
    commentstyle=\color{purple}\ttfamily,
    stringstyle=\color{red}\ttfamily
}

\lstset{
    language=Solidity,
    basicstyle=\small\ttfamily,
    numbers=left,
    numberstyle=\tiny,
    frame=single,
    breaklines=true,
    breakatwhitespace=false
}

% Hyperref setup
\hypersetup{
    colorlinks=true,
    linkcolor=blue,
    urlcolor=cyan,
    pdftitle={Analisi di Sicurezza DUAL-STRIDE-DUA},
    pdfauthor={Luca Belard},
}

\begin{document}

% ===== FRONTESPIZIO =====
\begin{titlepage}
    \centering
    \vspace*{2cm}
    {\LARGE\bfseries Università degli Studi\par}
    \vspace{0.5cm}
    {\Large Dipartimento di Informatica\par}
    \vspace{2cm}
    {\Huge\bfseries Analisi di Sicurezza DUAL-STRIDE-DUA\par}
    \vspace{0.5cm}
    {\Large Sistema Oracolo Bayesiano per la Catena del Freddo Farmaceutica\par}
    \vspace{2cm}
    {\large \textbf{Relazione Tecnica}\par}
    \vfill
    {\large \textbf{Autore:}\\ Luca Belard \par}
    \vspace{1cm}
    {\large 27 Gennaio 2026\par}
\end{titlepage}

\newpage
\tableofcontents
\newpage

% ===== CAPITOLO 1 =====
\section{Introduzione e Contesto}

Il presente documento descrive l'analisi di sicurezza effettuata sul sistema "Oracolo Bayesiano per la Catena del Freddo Farmaceutica". L'obiettivo è identificare, classificare e mitigare le vulnerabilità potenziali dell'architettura smart contract implementata su rete Ethereum/Hyperledger Besu.

La metodologia adottata, denominata **DUAL-STRIDE-DUA**, combina l'approccio classico STRIDE per la modellazione delle minacce intenzionali con l'analisi DUA (Defect Usage Analysis) per intercettare anche scenari di errore non intenzionale (misuse cases).

\subsection{Asset Critici del Sistema}
Il sistema gestisce asset di diversa natura, dai fondi in criptovaluta ai dati sensibili della rete bayesiana. 

\begin{table}[H]
\centering
\caption{Classificazione degli asset del sistema}
\label{tab:assets}
\rowcolors{2}{gray!10}{white}
\begin{tabular}{@{}llp{6cm}l@{}}
\toprule
\textbf{ID} & \textbf{Asset} & \textbf{Descrizione} & \textbf{Criticità} \\
\midrule
A1 & Smart Contract & Logica di business e fondi in escrow & Critica \\
A2 & Evidenze IoT & Dati dai sensori (E1-E5) & Critica \\
A3 & Pagamenti ETH & Fondi depositati dai mittenti & Critica \\
A4 & Ruoli e Permessi & Sistema AccessControl & Alta \\
A5 & CPT e Probabilità & Parametri della rete bayesiana & Alta \\
A6 & Dati spedizioni & Record on-chain e storico & Media \\
A7 & Interfaccia Web & Frontend utente (DApp) & Media \\
A8 & Chiavi private & Credenziali MetaMask & Critica \\
\bottomrule
\end{tabular}
\end{table}

% ===== CAPITOLO 2 =====
\newpage
\begin{landscape}
\section{Analisi STRIDE-DUA Integrata}

Di seguito viene presentata la matrice completa dell'analisi di rischio, che mappa gli asset contro le minacce STRIDE (intenzionali) e gli attributi di Safety/DUA (Danger, Unreliability, Absence of Resilience, Exposure). Gli identificativi di minaccia (es. S1.1, T1.1) verranno dettagliati nei capitoli successivi sia come casi di abuso (intenzionali) che di cattivo uso (accidentali).

\begin{longtable}{|p{1.5cm}|c|c|c|c|c|c|c|c|c|c|c|p{3cm}|c|c|p{3cm}|c|c|c|c|c|c|c|}
\hline
\rowcolor{headerblue}
\multicolumn{2}{|c|}{\textbf{\textcolor{white}{Asset}}} & \multicolumn{6}{c|}{\textbf{\textcolor{white}{STRIDE}}} & \multicolumn{4}{c|}{\textbf{\textcolor{white}{DUA (Safety)}}} & \multicolumn{1}{c|}{\textbf{\textcolor{white}{Attack}}} & \multicolumn{2}{c|}{\textbf{\textcolor{white}{Inherent}}} & \multicolumn{1}{c|}{\textbf{\textcolor{white}{Control}}} & \multicolumn{2}{c|}{\textbf{\textcolor{white}{Cost}}} & \multicolumn{3}{c|}{\textbf{\textcolor{white}{Residual}}} & \multicolumn{2}{c|}{\textbf{\textcolor{white}{Final}}} \\
\hline
\rowcolor{headerblue}
\tiny \textcolor{white}{Name} & \tiny \textcolor{white}{Val} & \tiny \textcolor{white}{S} & \tiny \textcolor{white}{T} & \tiny \textcolor{white}{R} & \tiny \textcolor{white}{I} & \tiny \textcolor{white}{D} & \tiny \textcolor{white}{E} & \tiny \textcolor{white}{Dan} & \tiny \textcolor{white}{Unr} & \tiny \textcolor{white}{AoR} & \tiny \textcolor{white}{Exp} & \tiny \textcolor{white}{Description} & \tiny \textcolor{white}{Prob} & \tiny \textcolor{white}{Risk} & \tiny \textcolor{white}{Mitigation} & \tiny \textcolor{white}{Cost} & \tiny \textcolor{white}{Feas} & \tiny \textcolor{white}{Prob} & \tiny \textcolor{white}{Imp} & \tiny \textcolor{white}{Risk} & \tiny \textcolor{white}{RoC} & \tiny \textcolor{white}{Tot} \\
\hline
\endfirsthead
\hline
\rowcolor{headerblue}
\tiny \textcolor{white}{Name} & \tiny \textcolor{white}{Val} & \tiny \textcolor{white}{S} & \tiny \textcolor{white}{T} & \tiny \textcolor{white}{R} & \tiny \textcolor{white}{I} & \tiny \textcolor{white}{D} & \tiny \textcolor{white}{E} & \tiny \textcolor{white}{Dan} & \tiny \textcolor{white}{Unr} & \tiny \textcolor{white}{AoR} & \tiny \textcolor{white}{Exp} & \tiny \textcolor{white}{Description} & \tiny \textcolor{white}{Prob} & \tiny \textcolor{white}{Risk} & \tiny \textcolor{white}{Mitigation} & \tiny \textcolor{white}{Cost} & \tiny \textcolor{white}{Feas} & \tiny \textcolor{white}{Prob} & \tiny \textcolor{white}{Imp} & \tiny \textcolor{white}{Risk} & \tiny \textcolor{white}{RoC} & \tiny \textcolor{white}{Tot} \\
\hline
\endhead

\tiny A2 & \tiny Crit & \cellcolor{red!30}X & & & & & & & \cellcolor{orange!30}X & & & \tiny Impersonificazione sensore (S1.1) & \tiny High & \tiny \textbf{Crit} & \tiny Role Check & \tiny Low & \tiny High & \tiny Low & \tiny Crit & \tiny Low & \tiny High & \tiny Low \\
\hline
\tiny A5 & \tiny High & & \cellcolor{red!30}X & & & & & \cellcolor{orange!30}X & & & & \tiny Manipolazione CPT (T1.1) & \tiny Med & \tiny \textbf{High} & \tiny Private Vars & \tiny Low & \tiny High & \tiny Low & \tiny High & \tiny Low & \tiny High & \tiny Low \\
\hline
\tiny A3 & \tiny Crit & & & & & \cellcolor{red!30}X & & & & \cellcolor{orange!30}X & & \tiny DoS Blocco Fondi (D1.2) & \tiny High & \tiny \textbf{Crit} & \tiny Refund Logic & \tiny Med & \tiny High & \tiny Low & \tiny Low & \tiny Low & \tiny High & \tiny Low \\
\hline
\tiny A6 & \tiny Med & & & & \cellcolor{red!30}X & & & & & & \cellcolor{orange!30}X & \tiny Exp. Dati Sensibili (I1.1) & \tiny High & \tiny Med & \tiny Hashing & \tiny Low & \tiny High & \tiny Low & \tiny Low & \tiny Low & \tiny Med & \tiny Low \\
\hline
\tiny A5 & \tiny High & & & & \cellcolor{red!30}X & & & & & \cellcolor{orange!30}X & \cellcolor{orange!30}X & \tiny Reverse Eng CPT (I1.2) & \tiny High & \tiny High & \tiny Private getter & \tiny Low & \tiny Med & \tiny Low & \tiny High & \tiny Med & \tiny High & \tiny Low \\
\hline
\tiny A7 & \tiny Med & \cellcolor{red!30}X & & & & & & \cellcolor{orange!30}X & & & & \tiny Spoofing UI (S1.2) & \tiny High & \tiny High & \tiny User Edu & \tiny High & \tiny Med & \tiny Med & \tiny High & \tiny Med & \tiny Low & \tiny High \\
\hline

\caption{Matrice Analisi Completa Dual-Stride}
\label{tab:dualstride}
\end{longtable}
\end{landscape}

\newpage

% ===== CAPITOLO 3: SCENARI DUALI =====
\section{Dettaglio Use/Misuse Cases per Threat ID}

In questa sezione, ogni minaccia identificata nella matrice STRIDE (CSE ID) viene esplosa nei suoi componenti Duali: uno scenario di abuso (intenzionale) e, ove applicabile, uno scenario di cattivo uso (accidentale/failure).

\subsection*{Threat ID: S1.1 - Impersonificazione Sensore}

% S1.1 Abuse
\begin{table}[H]
\centering
\renewcommand{\arraystretch}{1.3}
\begin{tabular}{|p{3cm}|p{12cm}|}
\hline
\rowcolor{headerblue}
\multicolumn{2}{|c|}{\textbf{\textcolor{white}{Case Identification}}} \\
\hline
\rowcolor{labelblue} \textbf{Case Type} & Abuse Case \\
\hline
\rowcolor{labelblue} \textbf{Case ID} & S1.1-A \\
\hline
\rowcolor{labelblue} \textbf{Case Name} & \textbf{Sensore Malevolo Falsificato} \\
\hline
\rowcolor{labelblue} \textbf{Actors} & Attaccante Esterno, Sensore Compromesso \\
\hline
\rowcolor{labelblue} \textbf{Description} & Un attaccante utilizza una chiave rubata con RUOLO\_SENSORE per inviare dati falsi (es. temperatura OK quando non lo è). \\
\hline
\rowcolor{labelblue} \textbf{Data} & Chiave privata, Dati sensori \\
\hline
\rowcolor{labelblue} \textbf{Basic Flow} & 
1. Attaccante compromette chiave sensore. \newline
2. Invia evidenza falsa $E_1=true$. \newline
3. Sistema valida spedizione non conforme. \\
\hline
\rowcolor{labelblue} \textbf{Response} & Pagamento fraudolento. \\
\hline
\end{tabular}
\end{table}

% S1.1 Misuse
\begin{table}[H]
\centering
\renewcommand{\arraystretch}{1.3}
\begin{tabular}{|p{3cm}|p{12cm}|}
\hline
\rowcolor{headerblue}
\multicolumn{2}{|c|}{\textbf{\textcolor{white}{Case Identification}}} \\
\hline
\rowcolor{labelblue} \textbf{Case Type} & Misuse Case \\
\hline
\rowcolor{labelblue} \textbf{Case ID} & S1.1-M \\
\hline
\rowcolor{labelblue} \textbf{Case Name} & \textbf{Guasto Sensore (Unreliability)} \\
\hline
\rowcolor{labelblue} \textbf{Actors} & Sensore Guasto \\
\hline
\rowcolor{labelblue} \textbf{Description} & Un sensore difettoso invia dati errati non per malizia ma per drift hw o rottura. \\
\hline
\rowcolor{labelblue} \textbf{Data} & Dati sensori \\
\hline
\rowcolor{labelblue} \textbf{Basic Flow} & 
1. Sensore starato legge +2 gradi. \newline
2. Invia $E_1=false$ erroneamente. \newline
3. Spedizione valida viene rifiutata. \\
\hline
\rowcolor{labelblue} \textbf{Response} & Falso negativo, mittente non pagato. \\
\hline
\end{tabular}
\end{table}

\subsection*{Threat ID: T1.1 - Manipolazione CPT}

% T1.1 Abuse
\begin{table}[H]
\centering
\renewcommand{\arraystretch}{1.3}
\begin{tabular}{|p{3cm}|p{12cm}|}
\hline
\rowcolor{headerblue}
\multicolumn{2}{|c|}{\textbf{\textcolor{white}{Case Identification}}} \\
\hline
\rowcolor{labelblue} \textbf{Case Type} & Abuse Case \\
\hline
\rowcolor{labelblue} \textbf{Case ID} & T1.1-A \\
\hline
\rowcolor{labelblue} \textbf{Case Name} & \textbf{Insider Manipulation CPT} \\
\hline
\rowcolor{labelblue} \textbf{Actors} & Admin Infedele \\
\hline
\rowcolor{labelblue} \textbf{Description} & Un admin con RUOLO\_ORACOLO modifica intenzionalmente le probabilità per favorire transazioni fraudolente. \\
\hline
\rowcolor{labelblue} \textbf{Data} & CPT Params \\
\hline
\rowcolor{labelblue} \textbf{Basic Flow} & 
1. Admin imposta $P(E|F)=99\%$ sempre. \newline
2. Qualsiasi evidenza viene accettata. \newline
3. Sistema logico compromesso. \\
\hline
\rowcolor{labelblue} \textbf{Response} & Validazione inutile. \\
\hline
\end{tabular}
\end{table}

% T1.1 Misuse
\begin{table}[H]
\centering
\renewcommand{\arraystretch}{1.3}
\begin{tabular}{|p{3cm}|p{12cm}|}
\hline
\rowcolor{headerblue}
\multicolumn{2}{|c|}{\textbf{\textcolor{white}{Case Identification}}} \\
\hline
\rowcolor{labelblue} \textbf{Case Type} & Misuse Case \\
\hline
\rowcolor{labelblue} \textbf{Case ID} & T1.1-M \\
\hline
\rowcolor{labelblue} \textbf{Case Name} & \textbf{Errore Configurazione CPT} \\
\hline
\rowcolor{labelblue} \textbf{Actors} & Admin Inesperto \\
\hline
\rowcolor{labelblue} \textbf{Description} & Inserimento accidentale di valori fuori range o invertiti (typo). \\
\hline
\rowcolor{labelblue} \textbf{Basic Flow} & 
1. Admin scrive 150 invece di 15. \newline
2. Calcolo produce overflow o errore logico. \\
\hline
\rowcolor{labelblue} \textbf{Response} & Sistema instabile fino a correzione. \\
\hline
\end{tabular}
\end{table}

\subsection*{Threat ID: D1.2 - Denial of Service (Blocco Fondi)}

% D1.2 Abuse
\begin{table}[H]
\centering
\renewcommand{\arraystretch}{1.3}
\begin{tabular}{|p{3cm}|p{12cm}|}
\hline
\rowcolor{headerblue}
\multicolumn{2}{|c|}{\textbf{\textcolor{white}{Case Identification}}} \\
\hline
\rowcolor{labelblue} \textbf{Case Type} & Abuse Case \\
\hline
\rowcolor{labelblue} \textbf{Case ID} & D1.2-A \\
\hline
\rowcolor{labelblue} \textbf{Case Name} & \textbf{Withholding Attack (Malevolo)} \\
\hline
\rowcolor{labelblue} \textbf{Actors} & Sensore Malevolo \\
\hline
\rowcolor{labelblue} \textbf{Description} & Sensore compromesso rifiuta selettivamente di inviare l'ultima evidenza per bloccare i fondi del mittente come riscatto. \\
\hline
\rowcolor{labelblue} \textbf{Basic Flow} & 
1. Invio E1-E4. \newline
2. Trattenuta E5. \newline
3. Fondi bloccati in escrow. \\
\hline
\rowcolor{labelblue} \textbf{Response} & Ricatto economico. \\
\hline
\end{tabular}
\end{table}

% D1.2 Misuse
\begin{table}[H]
\centering
\renewcommand{\arraystretch}{1.3}
\begin{tabular}{|p{3cm}|p{12cm}|}
\hline
\rowcolor{headerblue}
\multicolumn{2}{|c|}{\textbf{\textcolor{white}{Case Identification}}} \\
\hline
\rowcolor{labelblue} \textbf{Case Type} & Misuse Case \\
\hline
\rowcolor{labelblue} \textbf{Case ID} & D1.2-M \\
\hline
\rowcolor{labelblue} \textbf{Case Name} & \textbf{Timeout/Connettività (Accidentale)} \\
\hline
\rowcolor{labelblue} \textbf{Actors} & Sensore Offline \\
\hline
\rowcolor{labelblue} \textbf{Description} & Sensore perde connessione o batteria e non riesce a inviare evidenze. \\
\hline
\rowcolor{labelblue} \textbf{Basic Flow} & 
1. Spedizione parte. \newline
2. Sensore muore. \newline
3. Evidenze mai arrivate. \\
\hline
\rowcolor{labelblue} \textbf{Response} & Fondi bloccati per errore tecnico. \\
\hline
\end{tabular}
\end{table}

\subsection*{Threat ID: I1.1 - Esposizione Dati}

% I1.1 Abuse only (Misuse is rare for public ledger data)
\begin{table}[H]
\centering
\renewcommand{\arraystretch}{1.3}
\begin{tabular}{|p{3cm}|p{12cm}|}
\hline
\rowcolor{headerblue}
\multicolumn{2}{|c|}{\textbf{\textcolor{white}{Case Identification}}} \\
\hline
\rowcolor{labelblue} \textbf{Case Type} & Abuse Case \\
\hline
\rowcolor{labelblue} \textbf{Case ID} & I1.1-A \\
\hline
\rowcolor{labelblue} \textbf{Case Name} & \textbf{Data Mining Competitivo} \\
\hline
\rowcolor{labelblue} \textbf{Actors} & Competitor \\
\hline
\rowcolor{labelblue} \textbf{Description} & Analisi transazioni pubbliche per ricostruire volumi d'affari e clienti. \\
\hline
\rowcolor{labelblue} \textbf{Response} & Perdita segreto industriale. \\
\hline
\end{tabular}
\end{table}

\subsection*{Threat ID: S1.2 - Spoofing UI}

% S1.2 Abuse
\begin{table}[H]
\centering
\renewcommand{\arraystretch}{1.3}
\begin{tabular}{|p{3cm}|p{12cm}|}
\hline
\rowcolor{headerblue}
\multicolumn{2}{|c|}{\textbf{\textcolor{white}{Case Identification}}} \\
\hline
\rowcolor{labelblue} \textbf{Case Type} & Abuse Case \\
\hline
\rowcolor{labelblue} \textbf{Case ID} & S1.2-A \\
\hline
\rowcolor{labelblue} \textbf{Case Name} & \textbf{Phishing DApp} \\
\hline
\rowcolor{labelblue} \textbf{Actors} & Attaccante Phisher \\
\hline
\rowcolor{labelblue} \textbf{Description} & Cloning interfaccia web per rubare chiavi private agli utenti. \\
\hline
\rowcolor{labelblue} \textbf{Response} & Furto d'identità e fondi. \\
\hline
\end{tabular}
\end{table}

% S1.2 Misuse (User Error -> MC-03 equivalent)
\begin{table}[H]
\centering
\renewcommand{\arraystretch}{1.3}
\begin{tabular}{|p{3cm}|p{12cm}|}
\hline
\rowcolor{headerblue}
\multicolumn{2}{|c|}{\textbf{\textcolor{white}{Case Identification}}} \\
\hline
\rowcolor{labelblue} \textbf{Case Type} & Misuse Case \\
\hline
\rowcolor{labelblue} \textbf{Case ID} & S1.2-M \\
\hline
\rowcolor{labelblue} \textbf{Case Name} & \textbf{Errore Destinatario (User Mistake)} \\
\hline
\rowcolor{labelblue} \textbf{Actors} & Mittente Distratto \\
\hline
\rowcolor{labelblue} \textbf{Description} & Utente interagisce con contratto corretto ma invia pagamento a indirizzo sbagliato per typo. \\
\hline
\rowcolor{labelblue} \textbf{Response} & Perdita fondi irrevocabile. \\
\hline
\end{tabular}
\end{table}

% ===== CAPITOLO 4 =====
\section{Contromisure Implementate}

\subsection{Protezione dei Fondi (Anti-Blocco)}
Le contromisure rispondono direttamente ai casi D1.2-A e D1.2-M attraverso il meccanismo di rimborso a tre livelli.

\subsection{Privacy e Offuscamento Dati}
Le contromisure rispondono al caso I1.1-A (Hashing) e T1.1-A (CPT Private).

\section{Conclusioni}
L'applicazione sistematica dell'analisi Duale per ogni ID di minaccia (S1.1, T1.1, ecc.) ha garantito la copertura totale sia degli attacchi intenzionali che dei guasti accidentali.

\end{document}
