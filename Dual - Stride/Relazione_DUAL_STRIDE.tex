\documentclass[12pt,a4paper]{article}

% Pacchetti essenziali
\usepackage[utf8]{inputenc}
\usepackage[italian]{babel}
\usepackage[T1]{fontenc}
\usepackage{graphicx}
\usepackage{amsmath}
\usepackage{amssymb}
\usepackage{hyperref}
\usepackage{listings}
\usepackage[table]{xcolor}
\usepackage{colortbl}
\usepackage{geometry}
\usepackage{fancyhdr}
\usepackage{booktabs}
\usepackage{float}
\usepackage{caption}
\usepackage{subcaption}
\usepackage{pdflscape}
\usepackage{enumitem}
\usepackage{longtable}
\usepackage{multirow}

% Colori personalizzati per le tabelle
\definecolor{headerblue}{RGB}{51, 102, 204} % Blu scuro
\definecolor{labelblue}{RGB}{204, 229, 255}  % Blu chiaro

% Configurazione geometria pagina
\geometry{
    a4paper,
    left=2.5cm,
    right=2.5cm,
    top=2.5cm,
    bottom=2.5cm
}

% Configurazione header e footer
\pagestyle{fancy}
\fancyhf{}
\fancyhead[L]{\leftmark}
\fancyhead[R]{\thepage}
\renewcommand{\headrulewidth}{0.4pt}

% Code Listing Settings
\lstdefinelanguage{Solidity}{
    keywords={contract, function, public, private, external, internal, view, pure, payable, require, emit, event, struct, mapping, uint256, uint8, address, bool, bytes32, returns, if, else, return, storage, memory, calldata},
    keywordstyle=\color{blue}\bfseries,
    identifierstyle=\color{black},
    sensitive=false,
    comment=[l]{//},
    morecomment=[s]{/*}{*/},
    commentstyle=\color{purple}\ttfamily,
    stringstyle=\color{red}\ttfamily
}

\lstset{
    language=Solidity,
    basicstyle=\small\ttfamily,
    numbers=left,
    numberstyle=\tiny,
    frame=single,
    breaklines=true,
    breakatwhitespace=false
}

% Hyperref setup
\hypersetup{
    colorlinks=true,
    linkcolor=blue,
    urlcolor=cyan,
    pdftitle={Analisi di Sicurezza DUAL-STRIDE-DUA},
    pdfauthor={Luca Belard},
}

\begin{document}

% ===== FRONTESPIZIO =====
\begin{titlepage}
    \centering
    \vspace*{2cm}
    {\LARGE\bfseries Università degli Studi\par}
    \vspace{0.5cm}
    {\Large Dipartimento di Informatica\par}
    \vspace{2cm}
    {\Huge\bfseries Analisi di Sicurezza DUAL-STRIDE-DUA\par}
    \vspace{0.5cm}
    {\Large Sistema Oracolo Bayesiano per la Catena del Freddo Farmaceutica\par}
    \vspace{2cm}
    {\large \textbf{Relazione Tecnica}\par}
    \vfill
    {\large \textbf{Autore:}\\ Luca Belard \par}
    \vspace{1cm}
    {\large 27 Gennaio 2026\par}
\end{titlepage}

\newpage
\tableofcontents
\newpage

% ===== CAPITOLO 1 =====
\section{Introduzione e Contesto}

Il presente documento descrive l'analisi di sicurezza effettuata sul sistema "Oracolo Bayesiano per la Catena del Freddo Farmaceutica". L'obiettivo è identificare, classificare e mitigare le vulnerabilità potenziali dell'architettura smart contract implementata su rete Ethereum/Hyperledger Besu.

La metodologia adottata, denominata **DUAL-STRIDE-DUA**, combina l'approccio classico STRIDE per la modellazione delle minacce intenzionali con l'analisi DUA (Defect Usage Analysis) per intercettare anche scenari di errore non intenzionale (misuse cases).

\subsection{Asset Critici del Sistema}
Il sistema gestisce asset di diversa natura, dai fondi in criptovaluta ai dati sensibili della rete bayesiana. 

\begin{table}[H]
\centering
\caption{Classificazione degli asset del sistema}
\label{tab:assets}
\rowcolors{2}{gray!10}{white}
\begin{tabular}{@{}llp{6cm}l@{}}
\toprule
\textbf{ID} & \textbf{Asset} & \textbf{Descrizione} & \textbf{Criticità} \\
\midrule
A1 & Smart Contract & Logica di business e fondi in escrow & Critica \\
A2 & Evidenze IoT & Dati dai sensori (E1-E5) & Critica \\
A3 & Pagamenti ETH & Fondi depositati dai mittenti & Critica \\
A4 & Ruoli e Permessi & Sistema AccessControl & Alta \\
A5 & CPT e Probabilità & Parametri della rete bayesiana & Alta \\
A6 & Dati spedizioni & Record on-chain e storico & Media \\
A7 & Interfaccia Web & Frontend utente (DApp) & Media \\
A8 & Chiavi private & Credenziali MetaMask & Critica \\
\bottomrule
\end{tabular}
\end{table}

% ===== CAPITOLO 2 =====
\newpage
\begin{landscape}
\section{Analisi STRIDE-DUA Integrata}

Di seguito viene presentata la matrice completa dell'analisi di rischio, che mappa gli asset contro le minacce STRIDE (intenzionali) e gli attributi di Safety/DUA (Danger, Unreliability, Absence of Resilience, Exposure).

\begin{longtable}{|p{1.5cm}|c|c|c|c|c|c|c|c|c|c|c|p{3cm}|c|c|p{3cm}|c|c|c|c|c|c|c|}
\hline
\rowcolor{headerblue}
\multicolumn{2}{|c|}{\textbf{\textcolor{white}{Asset}}} & \multicolumn{6}{c|}{\textbf{\textcolor{white}{STRIDE}}} & \multicolumn{4}{c|}{\textbf{\textcolor{white}{DUA (Safety)}}} & \multicolumn{1}{c|}{\textbf{\textcolor{white}{Attack}}} & \multicolumn{2}{c|}{\textbf{\textcolor{white}{Inherent}}} & \multicolumn{1}{c|}{\textbf{\textcolor{white}{Control}}} & \multicolumn{2}{c|}{\textbf{\textcolor{white}{Cost}}} & \multicolumn{3}{c|}{\textbf{\textcolor{white}{Residual}}} & \multicolumn{2}{c|}{\textbf{\textcolor{white}{Final}}} \\
\hline
\rowcolor{headerblue}
\tiny \textcolor{white}{Name} & \tiny \textcolor{white}{Val} & \tiny \textcolor{white}{S} & \tiny \textcolor{white}{T} & \tiny \textcolor{white}{R} & \tiny \textcolor{white}{I} & \tiny \textcolor{white}{D} & \tiny \textcolor{white}{E} & \tiny \textcolor{white}{Dan} & \tiny \textcolor{white}{Unr} & \tiny \textcolor{white}{AoR} & \tiny \textcolor{white}{Exp} & \tiny \textcolor{white}{Description} & \tiny \textcolor{white}{Prob} & \tiny \textcolor{white}{Risk} & \tiny \textcolor{white}{Mitigation} & \tiny \textcolor{white}{Cost} & \tiny \textcolor{white}{Feas} & \tiny \textcolor{white}{Prob} & \tiny \textcolor{white}{Imp} & \tiny \textcolor{white}{Risk} & \tiny \textcolor{white}{RoC} & \tiny \textcolor{white}{Tot} \\
\hline
\endfirsthead
\hline
\rowcolor{headerblue}
\tiny \textcolor{white}{Name} & \tiny \textcolor{white}{Val} & \tiny \textcolor{white}{S} & \tiny \textcolor{white}{T} & \tiny \textcolor{white}{R} & \tiny \textcolor{white}{I} & \tiny \textcolor{white}{D} & \tiny \textcolor{white}{E} & \tiny \textcolor{white}{Dan} & \tiny \textcolor{white}{Unr} & \tiny \textcolor{white}{AoR} & \tiny \textcolor{white}{Exp} & \tiny \textcolor{white}{Description} & \tiny \textcolor{white}{Prob} & \tiny \textcolor{white}{Risk} & \tiny \textcolor{white}{Mitigation} & \tiny \textcolor{white}{Cost} & \tiny \textcolor{white}{Feas} & \tiny \textcolor{white}{Prob} & \tiny \textcolor{white}{Imp} & \tiny \textcolor{white}{Risk} & \tiny \textcolor{white}{RoC} & \tiny \textcolor{white}{Tot} \\
\hline
\endhead

\tiny A2 & \tiny Crit & \cellcolor{red!30}X & & & & & & & \cellcolor{orange!30}X & & & \tiny Impersonificazione sensore (S1.1) & \tiny High & \tiny \textbf{Crit} & \tiny Role Check & \tiny Low & \tiny High & \tiny Low & \tiny Crit & \tiny Low & \tiny High & \tiny Low \\
\hline
\tiny A5 & \tiny High & & \cellcolor{red!30}X & & & & & \cellcolor{orange!30}X & & & & \tiny Manipolazione CPT (T1.1) & \tiny Med & \tiny \textbf{High} & \tiny Private Vars & \tiny Low & \tiny High & \tiny Low & \tiny High & \tiny Low & \tiny High & \tiny Low \\
\hline
\tiny A3 & \tiny Crit & & & & & \cellcolor{red!30}X & & & & \cellcolor{orange!30}X & & \tiny DoS Blocco Fondi (D1.2) & \tiny High & \tiny \textbf{Crit} & \tiny Refund Logic & \tiny Med & \tiny High & \tiny Low & \tiny Low & \tiny Low & \tiny High & \tiny Low \\
\hline
\tiny A6 & \tiny Med & & & & \cellcolor{red!30}X & & & & & & \cellcolor{orange!30}X & \tiny Exp. Dati Sensibili (I1.1) & \tiny High & \tiny Med & \tiny Hashing & \tiny Low & \tiny High & \tiny Low & \tiny Low & \tiny Low & \tiny Med & \tiny Low \\
\hline
\tiny A5 & \tiny High & & & & \cellcolor{red!30}X & & & & & \cellcolor{orange!30}X & \cellcolor{orange!30}X & \tiny Reverse Eng CPT (I1.2) & \tiny High & \tiny High & \tiny Private getter & \tiny Low & \tiny Med & \tiny Low & \tiny High & \tiny Med & \tiny High & \tiny Low \\
\hline
\tiny A7 & \tiny Med & \cellcolor{red!30}X & & & & & & \cellcolor{orange!30}X & & & & \tiny Spoofing UI (S1.2) & \tiny High & \tiny High & \tiny User Edu & \tiny High & \tiny Med & \tiny Med & \tiny High & \tiny Med & \tiny Low & \tiny High \\
\hline

\caption{Matrice Analisi Completa Dual-Stride}
\label{tab:dualstride}
\end{longtable}
\end{landscape}

\newpage

% ===== CAPITOLO 3: SCENARI =====
\section{Use Case e Misuse Case}

\subsection{Use Cases (Attacchi Intenzionali)}

% UC-01
\begin{table}[H]
\centering
\renewcommand{\arraystretch}{1.3}
\begin{tabular}{|p{3cm}|p{12cm}|}
\hline
\rowcolor{headerblue}
\multicolumn{2}{|c|}{\textbf{\textcolor{white}{Case Identification}}} \\
\hline
\rowcolor{labelblue} \textbf{Case Type} & Use Case \\
\hline
\rowcolor{labelblue} \textbf{Case ID} & UC-01 \\
\hline
\rowcolor{labelblue} \textbf{Case Name} & \textbf{Manipolazione Evidenze Sensore} \\
\hline
\rowcolor{labelblue} \textbf{Actors} & Attaccante Esterno, Sensore Compromesso, Corriere Complice \\
\hline
\rowcolor{labelblue} \textbf{Description} & Un attaccante invia evidenze falsificate sfruttando una chiave privata compromessa per ottenere indebitamente il pagamento. \\
\hline
\rowcolor{labelblue} \textbf{Data} & Chiave privata, Dati sensori (E1-E5) \\
\hline
\rowcolor{labelblue} \textbf{Stimulus/Precond.} & Sensore con ruolo attivo; spedizione in stato InAttesa. \\
\hline
\rowcolor{labelblue} \textbf{Basic Flow} & 
1. Attaccante ottiene chiave privata sensore. \newline
2. Crea spedizione fraudolenta. \newline
3. Invia $E_1=true$ (falso) allo smart contract. \newline
4. Sistema calcola probabilità > 95\%. \newline
5. Corriere incassa pagamento. \\
\hline
\rowcolor{labelblue} \textbf{Alternative Flow} & Attacco Man-in-the-Middle su device IoT non protetto. \\
\hline
\rowcolor{labelblue} \textbf{Exception Flow} & Admin revoca ruolo DOPO invio -> Transazione fallisce. \\
\hline
\rowcolor{labelblue} \textbf{Response/Post.} & Perdita fondi, farmaci deteriorati in commercio. \\
\hline
\rowcolor{labelblue} \textbf{Non-Funct. Req.} & Autenticazione hardware (TPM). \\
\hline
\rowcolor{labelblue} \textbf{Comments} & Rischio Critico. Richiede device attestation. \\
\hline
\end{tabular}
\end{table}

% UC-02
\begin{table}[H]
\centering
\renewcommand{\arraystretch}{1.3}
\begin{tabular}{|p{3cm}|p{12cm}|}
\hline
\rowcolor{headerblue}
\multicolumn{2}{|c|}{\textbf{\textcolor{white}{Case Identification}}} \\
\hline
\rowcolor{labelblue} \textbf{Case Type} & Use Case \\
\hline
\rowcolor{labelblue} \textbf{Case ID} & UC-02 \\
\hline
\rowcolor{labelblue} \textbf{Case Name} & \textbf{Reverse Engineering CPT} \\
\hline
\rowcolor{labelblue} \textbf{Actors} & Attaccante Esterno con competenze tecniche \\
\hline
\rowcolor{labelblue} \textbf{Description} & Attaccante analizza le CPT (Conditional Probability Tables) per determinare le combinazioni minime di evidenze necessarie, permettendo di aggirare i controlli. \\
\hline
\rowcolor{labelblue} \textbf{Data} & CPT per E1-E5 (se pubbliche), probabilità P(F1), P(F2) \\
\hline
\rowcolor{labelblue} \textbf{Stimulus/Precond.} & Attaccante ha accesso lettura blockchain; CPT leggibili. \\
\hline
\rowcolor{labelblue} \textbf{Basic Flow} & 
1. Attaccante legge CPT dalla blockchain. \newline
2. Implementa algoritmo bayesiano offline. \newline
3. Simula tutte le 32 combinazioni possibili. \newline
4. Identifica combinazioni "economiche" che superano la soglia. \newline
5. Programma sensori per inviare esattamente quei dati. \\
\hline
\rowcolor{labelblue} \textbf{Alternative Flow} & Utilizzo di Machine Learning per inferire parametri da osservazioni passate. \\
\hline
\rowcolor{labelblue} \textbf{Response/Post.} & Spese minime per l'attaccante, validazione garantita fraudolenta. \\
\hline
\rowcolor{labelblue} \textbf{Comments} & \textbf{MITIGATO}: Variabili CPT rese private e accessibili solo agli admin. \\
\hline
\end{tabular}
\end{table}

% UC-03
\begin{table}[H]
\centering
\renewcommand{\arraystretch}{1.3}
\begin{tabular}{|p{3cm}|p{12cm}|}
\hline
\rowcolor{headerblue}
\multicolumn{2}{|c|}{\textbf{\textcolor{white}{Case Identification}}} \\
\hline
\rowcolor{labelblue} \textbf{Case Type} & Use Case \\
\hline
\rowcolor{labelblue} \textbf{Case ID} & UC-03 \\
\hline
\rowcolor{labelblue} \textbf{Case Name} & \textbf{DoS via Evidenze Mancanti} \\
\hline
\rowcolor{labelblue} \textbf{Actors} & Sensore Malevolo, Insider \\
\hline
\rowcolor{labelblue} \textbf{Description} & Un sensore non invia tutte le evidenze richieste, causando il blocco indefinito dei fondi in escrow. \\
\hline
\rowcolor{labelblue} \textbf{Data} & ID spedizione, ETH bloccati \\
\hline
\rowcolor{labelblue} \textbf{Stimulus/Precond.} & Spedizione creata con ETH in escrow; sensore ha RUOLO\_SENSORE. \\
\hline
\rowcolor{labelblue} \textbf{Basic Flow} & 
1. Mittente deposita 10 ETH. \newline
2. Sensore invia parzialmente (es. E1-E4) ma omette E5. \newline
3. Funzione \code{validaEPaga()} fallisce ("Evidenze mancanti"). \newline
4. Fondi bloccati permanentemente (senza correzioni). \\
\hline
\rowcolor{labelblue} \textbf{Alternative Flow} & Attacco distribuito su più spedizioni contemporaneamente. \\
\hline
\rowcolor{labelblue} \textbf{Response/Post.} & Blocco operativo e finanziario. \\
\hline
\rowcolor{labelblue} \textbf{Comments} & \textbf{RISOLTO}: Timeout rimborso 7 giorni e annullamento precoce. \\
\hline
\end{tabular}
\end{table}

% UC-04 (Restored)
\begin{table}[H]
\centering
\renewcommand{\arraystretch}{1.3}
\begin{tabular}{|p{3cm}|p{12cm}|}
\hline
\rowcolor{headerblue}
\multicolumn{2}{|c|}{\textbf{\textcolor{white}{Case Identification}}} \\
\hline
\rowcolor{labelblue} \textbf{Case Type} & Use Case \\
\hline
\rowcolor{labelblue} \textbf{Case ID} & UC-04 \\
\hline
\rowcolor{labelblue} \textbf{Case Name} & \textbf{Phishing tramite Clone Interfaccia} \\
\hline
\rowcolor{labelblue} \textbf{Actors} & Attaccante Esterno \\
\hline
\rowcolor{labelblue} \textbf{Description} & L'attaccante clona la DApp su un dominio simile per sottrarre chiavi private. \\
\hline
\rowcolor{labelblue} \textbf{Data} & Interfaccia Web clone, Credenziali utente \\
\hline
\rowcolor{labelblue} \textbf{Stimulus/Precond.} & Utente accede a dominio typosquatted. \\
\hline
\rowcolor{labelblue} \textbf{Basic Flow} & 
1. Attaccante registra dominio simile. \newline
2. Clona DApp React. \newline
3. Sito chiede Seed Phrase "per verifica". \newline
4. Utente inserisce credenziali. \newline
5. Furto wallet. \\
\hline
\rowcolor{labelblue} \textbf{Response/Post.} & Compromissione totale account. \\
\hline
\rowcolor{labelblue} \textbf{Comments} & Mitigazione: ENS, verifica IPFS hash. \\
\hline
\end{tabular}
\end{table}

% UC-05 (Restored)
\begin{table}[H]
\centering
\renewcommand{\arraystretch}{1.3}
\begin{tabular}{|p{3cm}|p{12cm}|}
\hline
\rowcolor{headerblue}
\multicolumn{2}{|c|}{\textbf{\textcolor{white}{Case Identification}}} \\
\hline
\rowcolor{labelblue} \textbf{Case Type} & Use Case \\
\hline
\rowcolor{labelblue} \textbf{Case ID} & UC-05 \\
\hline
\rowcolor{labelblue} \textbf{Case Name} & \textbf{Intelligence Competitiva (Data Mining)} \\
\hline
\rowcolor{labelblue} \textbf{Actors} & Competitor Commerciale \\
\hline
\rowcolor{labelblue} \textbf{Description} & Monitoraggio blockchain per ricostruire volumi e clienti del rivale. \\
\hline
\rowcolor{labelblue} \textbf{Data} & Transazioni pubbliche, Log eventi \\
\hline
\rowcolor{labelblue} \textbf{Stimulus/Precond.} & Dati in chiaro on-chain. \\
\hline
\rowcolor{labelblue} \textbf{Basic Flow} & 
1. Ascolto eventi \code{SpedizioneCreata}. \newline
2. Analisi frequenza e importi ETH. \newline
3. Mappatura grafo mittente-corriere. \newline
4. Deduzione strategia commerciale. \\
\hline
\rowcolor{labelblue} \textbf{Response/Post.} & Perdita segreto commerciale. \\
\hline
\rowcolor{labelblue} \textbf{Comments} & \textbf{MITIGATO}: Hashing dettagli on-chain. \\
\hline
\end{tabular}
\end{table}


\subsection{Misuse Cases (Errori Non Intenzionali)}

% MC-01
\begin{table}[H]
\centering
\renewcommand{\arraystretch}{1.3}
\begin{tabular}{|p{3cm}|p{12cm}|}
\hline
\rowcolor{headerblue}
\multicolumn{2}{|c|}{\textbf{\textcolor{white}{Case Identification}}} \\
\hline
\rowcolor{labelblue} \textbf{Case Type} & Misuse Case \\
\hline
\rowcolor{labelblue} \textbf{Case ID} & MC-01 \\
\hline
\rowcolor{labelblue} \textbf{Case Name} & \textbf{Smarrimento Chiavi Private} \\
\hline
\rowcolor{labelblue} \textbf{Actors} & Mittente Inesperto \\
\hline
\rowcolor{labelblue} \textbf{Description} & Un utente legittimo perde accesso alla propria chiave privata MetaMask. \\
\hline
\rowcolor{labelblue} \textbf{Data} & Chiave privata, Seed phrase \\
\hline
\rowcolor{labelblue} \textbf{Stimulus/Precond.} & Errore umano; nessun backup. \\
\hline
\rowcolor{labelblue} \textbf{Basic Flow} & 
1. Utente perde dispositivo/seed. \newline
2. Spedizione in corso richiede interazione (es. rimborso). \newline
3. Utente non può firmare. \newline
4. Fondi inaccessibili. \\
\hline
\rowcolor{labelblue} \textbf{Response/Post.} & Perdita permanente fondi. \\
\hline
\rowcolor{labelblue} \textbf{Comments} & Problema infrastrutturale. \\
\hline
\end{tabular}
\end{table}

% MC-02
\begin{table}[H]
\centering
\renewcommand{\arraystretch}{1.3}
\begin{tabular}{|p{3cm}|p{12cm}|}
\hline
\rowcolor{headerblue}
\multicolumn{2}{|c|}{\textbf{\textcolor{white}{Case Identification}}} \\
\hline
\rowcolor{labelblue} \textbf{Case Type} & Misuse Case \\
\hline
\rowcolor{labelblue} \textbf{Case ID} & MC-02 \\
\hline
\rowcolor{labelblue} \textbf{Case Name} & \textbf{Errore Configurazione Parametri} \\
\hline
\rowcolor{labelblue} \textbf{Actors} & Amministratore (Oracolo) \\
\hline
\rowcolor{labelblue} \textbf{Description} & Inserimento valori probabilistici errati (es. > 100\%). \\
\hline
\rowcolor{labelblue} \textbf{Data} & CPT params \\
\hline
\rowcolor{labelblue} \textbf{Stimulus/Precond.} & Typo o incomprensione unità. \\
\hline
\rowcolor{labelblue} \textbf{Basic Flow} & 
1. Admin chiama \code{impostaCPT} con 150. \newline
2. Sistema (senza fix) accetterebbe valore. \newline
3. Logica corrotta. \\
\hline
\rowcolor{labelblue} \textbf{Exception Flow} & Smart contract revert per check \code{<= 100}. \\
\hline
\rowcolor{labelblue} \textbf{Response/Post.} & Transazione fallita (con fix), sistema salvo. \\
\hline
\rowcolor{labelblue} \textbf{Comments} & \textbf{RISOLTO}: Controlli di validazione input. \\
\hline
\end{tabular}
\end{table}

% MC-03 (Restored)
\begin{table}[H]
\centering
\renewcommand{\arraystretch}{1.3}
\begin{tabular}{|p{3cm}|p{12cm}|}
\hline
\rowcolor{headerblue}
\multicolumn{2}{|c|}{\textbf{\textcolor{white}{Case Identification}}} \\
\hline
\rowcolor{labelblue} \textbf{Case Type} & Misuse Case \\
\hline
\rowcolor{labelblue} \textbf{Case ID} & MC-03 \\
\hline
\rowcolor{labelblue} \textbf{Case Name} & \textbf{Invio ETH Indirizzo Errato} \\
\hline
\rowcolor{labelblue} \textbf{Actors} & Mittente \\
\hline
\rowcolor{labelblue} \textbf{Description} & Mittente specifica indirizzo corriere errato (typo) in creazione. \\
\hline
\rowcolor{labelblue} \textbf{Data} & Indirizzo corriere, ETH \\
\hline
\rowcolor{labelblue} \textbf{Stimulus/Precond.} & Errore di copia-incolla. \\
\hline
\rowcolor{labelblue} \textbf{Basic Flow} & 
1. Copia indirizzo errato (0xABD...). \newline
2. Invoca \code{creaSpedizione}. \newline
3. Validazione OK. \newline
4. Pagamento a sconosciuto. \\
\hline
\rowcolor{labelblue} \textbf{Response/Post.} & Fondi persi a destinatario errato. \\
\hline
\rowcolor{labelblue} \textbf{Comments} & Mitigazione: Whitelist UI, Checksum address. \\
\hline
\end{tabular}
\end{table}


% ===== CAPITOLO 4 =====
\section{Contromisure Implementate}

\subsection{Protezione dei Fondi (Anti-Blocco)}
Per mitigare il rischio critico di blocco dei fondi (UC-03), è stato implementato un meccanismo di rimborso a tre livelli nel contratto \code{BNGestoreSpedizioni}:

\begin{enumerate}
    \item \textbf{Annullamento Precoce}: Permette al mittente di annullare la spedizione e recuperare i fondi \textit{prima} che qualsiasi evidenza venga inviata.
    \item \textbf{Timeout Temporale}: Se le evidenze non sono complete entro 7 giorni, o se la validazione non avviene entro 14 giorni, il rimborso è sbloccato.
    \item \textbf{Soglia Fallimenti}: Dopo 3 tentativi di validazione falliti (indicativi di merce non conforme), il mittente può richiedere indietro i fondi.
\end{enumerate}

\subsection{Privacy e Offuscamento Dati}
Per proteggere i segreti commerciali, sono state adottate strategie di minimizzazione dati on-chain.

\paragraph{Offuscamento Dettagli Spedizione}
La funzione \code{creaSpedizioneConHash} registra on-chain solo l'hash (Keccak-256) dei dettagli sensibili (merce, destinatario finale). I dati in chiaro rimangono off-chain, ma la loro integrità è verificabile in caso di disputa tramite \code{verificaDettagli}.

\paragraph{Visibilità CPT}
Le tabelle di probabilità condizionata sono state dichiarate \code{private} nel contratto \code{BNCore}. Questo impedisce la lettura diretta tramite getter automatici, alzando l'asticella per gli attaccanti che volessero fare reverse-engineering del modello (UC-02).

\section{Conclusioni e Sviluppi Futuri}
L'analisi Dual-Stride ha permesso di identificare e risolvere le criticità maggiori. Il sistema v3.0 presenta un profilo di sicurezza notevolmente rafforzato rispetto alle versioni precedenti.

L'adozione di un approccio metodologico strutturato ha permesso di coprire non solo gli attacchi intenzionali classici, ma anche scenari di errore umano spesso trascurati in ambito blockchain. Le contromisure implementate (rimborso, hashing, private variables) sono efficaci e verificabili direttamente nel codice sorgente.

\end{document}
