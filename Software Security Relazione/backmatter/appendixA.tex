\chapter{Guida al Deployment e Management}
\label{app:deployment}

\section{Requisiti di Sistema (Prerequisiti)}
Per eseguire l'intero stack del progetto sono necessari i seguenti strumenti:
\begin{itemize}
    \item \textbf{Node.js} (versione $\ge$ 16.0.0) e \textbf{NPM}
    \item \textbf{Docker} e \textbf{Docker Compose} (per il nodo Besu)
    \item \textbf{Truffle Suite} (per compilazione e deploy Smart Contracts)
    \item \textbf{Ganache} (opzionale, per test rapidi in locale)
    \item \textbf{Git}
    \item \textbf{Java JDK 11+} (se si esegue Besu nativamente senza Docker)
\end{itemize}

\section{Installazione e Setup}
\subsection{Clonazione del Repository}
Il codice sorgente è ospitato su GitHub. Eseguire il clone:
\begin{lstlisting}[language=bash]
git clone https://github.com/lucabelard/ProgettoSoftwareSecurity.git
cd ProgettoSoftwareSecurity
\end{lstlisting}

\subsection{Installazione Dipendenze}
Installare le dipendenze per l'interfaccia web e gli script di test:
\begin{lstlisting}[language=bash]
npm install
cd web-interface
npm install
\end{lstlisting}

\section{Avvio della Rete Blockchain}
\subsection{Modalità Sviluppo (Ganache)}
1. Avviare Ganache (GUI o CLI) sulla porta 7545.
2. Configurare \texttt{truffle-config.js} per puntare a \texttt{127.0.0.1:7545}.
3. Eseguire il deploy:
\begin{lstlisting}[language=bash]
truffle migrate --reset --network development
\end{lstlisting}

\subsection{Modalità Produzione (Hyperledger Besu)}
1. Navigare nella cartella \texttt{besu-network}.
2. Avviare i nodi validatori:
\begin{lstlisting}[language=bash]
./start_nodes.sh
\end{lstlisting}
3. Attendere che i nodi siano sincronizzati (consenso IBFT 2.0).
4. Eseguire il deploy sulla rete Besu:
\begin{lstlisting}[language=bash]
truffle migrate --reset --network besu
\end{lstlisting}

\section{Esecuzione degli Script di Simulazione}
Per testare il sistema end-to-end con dati simulati:
\begin{lstlisting}[language=bash]
node simula_oracolo.js
\end{lstlisting}
Questo script simulerà l'invio di dati dai sensori e l'interazione con l'Oracolo on-chain.

\section{Interfaccia Web}
Per avviare la dashboard utente (necessita di Node.js installato):
\begin{lstlisting}[language=bash]
cd web-interface
npx http-server .
\end{lstlisting}
L'applicazione sarà accessibile di default a \texttt{http://localhost:8080}.
Assicurarsi di avere MetaMask configurato sulla rete locale (Chain ID 1337 o 2024 a seconda della configurazione Ganache/Besu).
