\chapter{Analisi delle Performance e Scalabilità}
\label{app:performance}

In questa appendice vengono presentati i risultati dei test di performance condotti per valutare la scalabilità del sistema proposto. L'obiettivo dell'analisi è quantificare il consumo di risorse, in termini di Gas e tempo di esecuzione, al crescere della complessità della Rete Bayesiana gestita dallo smart contract.
    
    \section{Metodologia di Test e Scenari}
    
    L'analisi è stata condotta su un'istanza locale di \textit{Hyperledger Besu}, configurata come rete privata per garantire un ambiente controllato e riproducibile. La procedura di test è stata interamente automatizzata attraverso una suite di script sviluppati ad hoc. In particolare, l'esecuzione delle interazioni con la blockchain è stata gestita da script JavaScript (\textit{test/performance/simple-performance-test.js}) che utilizzano la libreria \textit{web3.js} per invocare le funzioni degli smart contract e misurarne i consumi.
    
    Per simulare scenari di utilizzo realistici e valutare la risposta del sistema a carichi crescenti, sono state realizzate tre specifiche varianti del contratto principale, situate nella cartella \textit{contracts/performance}. Queste varianti differiscono per la complessità del grafo probabilistico implementato:
    
    \begin{itemize}
        \item \textit{BN\_Simple}: rappresenta lo scenario base con 1 solo nodo Fatto (radice) e 2 nodi Evidenza. È il caso minimalista utile come baseline.
        \item \textit{BN\_Medium}: introduce una complessità intermedia con 2 nodi Fatto e 3 nodi Evidenza, aumentando le interdipendenze.
        \item \textit{BN\_Complex}: è lo scenario più articolato, con 2 nodi Fatto e 5 nodi Evidenza. Questo contratto stressa maggiormente la logica di calcolo on-chain, simulando un monitoraggio completo della spedizione.
    \end{itemize}
    
    I dati grezzi raccolti durante le esecuzioni (consumo di gas per deploy, configurazione e validazione, oltre ai tempi di risposta) sono stati salvati in formato CSV. Successivamente, la generazione dei grafici e l'analisi statistica sono state affidate allo script Python \textit{test/performance/generate\_graphs.py}, che ha elaborato i dataset per produrre le visualizzazioni incluse in questa appendice.
    
    \section{Analisi dei Risultati}
    
    L'aspetto più critico per la sostenibilità economica e tecnica di una soluzione blockchain è il consumo di Gas. Il grafico in Figura \ref{fig:gas_consumption} illustra chiaramente la distinzione tra i costi "una tantum" di configurazione e i costi operativi ricorrenti.
    
    \begin{figure}[ht!]
        \centering
        \includegraphics[width=0.9\linewidth]{backmatter/image/gas_consumption.png}
        \caption{Confronto dei costi operativi in Gas: Configurazione vs Inferenza}
        \label{fig:gas_consumption}
    \end{figure}
    
    Si può osservare come il costo di \textit{Configurazione} (barre blu) cresca in modo lineare rispetto alla complessità della rete. Questo comportamento è atteso, poiché ogni nuovo nodo inserito nel modello richiede transazioni aggiuntive per salvare le relative Tabelle di Probabilità Condizionata (CPT) nello storage persistente della blockchain. Tuttavia, è fondamentale notare che questo costo viene sostenuto una sola volta, nella fase di inizializzazione del sistema.
    
    Al contrario, il costo di \textit{Inferenza} (barre verdi), che rappresenta la spesa per ogni singola validazione di spedizione, si mantiene sorprendentemente contenuto. Anche nello scenario più complesso (\textit{BN\_Complex}), il consumo di gas rimane ben al di sotto delle 80.000 unità. Questo dato è molto positivo, in quanto dimostra che l'algoritmo di propagazione della credenza implementato in Solidity è efficiente e che l'aumento dei nodi non provoca un'esplosione esponenziale dei costi di calcolo per transazione.
    
    Passando all'analisi temporale, la Figura \ref{fig:execution_time} riporta i tempi di esecuzione per le diverse fasi, utilizzando una scala logaritmica per apprezzare le differenze di ordine di grandezza.
    
    \begin{figure}[ht!]
        \centering
        \includegraphics[width=0.9\linewidth]{backmatter/image/execution_time.png}
        \caption{Tempi di esecuzione (Scala Logaritmica)}
        \label{fig:execution_time}
    \end{figure}
    
    Il grafico evidenzia come il tempo necessario per la validazione di una transazione (linea verde) sia trascurabile, attestandosi su valori nell'ordine delle decine di millisecondi. I tempi più elevati si registrano solo durante il \textit{Deploy} e la \textit{Configurazione} iniziale, operazioni che, come già sottolineato, non impattano sull'operatività quotidiana della piattaforma. Questo conferma l'idoneità dell'architettura per scenari real-time o near real-time.
    
    Infine, è stato analizzato l'impatto della logica aggiuntiva sulla dimensione del contratto compilato, come mostrato in Figura \ref{fig:contract_size}.
    
    \begin{figure}[ht!]
        \centering
        \includegraphics[width=0.8\linewidth]{backmatter/image/contract_size.png}
        \caption{Dimensione del Bytecode dello Smart Contract}
        \label{fig:contract_size}
    \end{figure}
    
    Anche in questo caso la crescita è moderata e lineare. Il contratto più complesso occupa meno di 8 KB di spazio, un valore ben lontano dal limite massimo di 24 KB imposto dallo standard EIP-170 ("Spurious Dragon"). Ciò garantisce un ampio margine per future estensioni delle funzionalità o per l'implementazione di reti bayesiane ancora più articolate senza rischiare di superare i limiti fisici del protocollo Ethereum.
    
    In conclusione, i test effettuati confermano la validità delle scelte progettuali. L'approccio on-chain per la logica bayesiana si dimostra non solo tecnicamente fattibile, ma anche scalabile ed efficiente per i casi d'uso previsti nella supply chain.
