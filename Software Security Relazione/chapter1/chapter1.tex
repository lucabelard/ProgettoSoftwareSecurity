\chapter{Valutazione del Rischio}

\begin{preamble}
In questo capitolo viene presentata un'analisi approfondita della sicurezza del sistema, condotta attraverso metodologie formali e strutturate. Il percorso logico inizia con la modellazione degli obiettivi e delle dipendenze strategiche tramite framework \textbf{i* (iStar)}, prosegue con la valutazione delle minacce mediante approccio \textbf{DUAL-STRIDE} esteso agli asset dell'attore sistema, e si conclude con la definizione dettagliata di scenari di \textbf{Abuse e Misuse Cases}.
\end{preamble}

\section{Modellazione Strategica e Attori}
Per comprendere appieno il contesto organizzativo e tecnico, l'analisi parte dalla realizzazione di modelli i* che evidenziano le relazioni tra attori, obiettivi (Goals), compiti (Tasks) e risorse (Resources).

Il primo passo è stato analizzare la **Supply Chain As-Is** (senza sistema blockchain). Come mostrato nella Figura \ref{fig:sd_as_is}, il modello Strategic Dependency (SD) evidenzia una criticità fondamentale: esistono dipendenze di "fiducia cieca" tra Produttore, Distributore e Farmacia. Il Produttore dipende dal Distributore per la corretta conservazione, ma non possiede mezzi diretti per verificarne l'operato, generando un softgoal "integrità non verificabile".

\begin{figure}[ht!]
    \centering
    \includegraphics[width=\linewidth, keepaspectratio]{chapter1/image/sd_as_is.png}
    \caption{Modello SD/SR senza sistema}
    \label{fig:sd_as_is}
\end{figure}

L'introduzione del sistema modifica radicalmente queste dinamiche, creando lo scenario **To-Be**. L'ingresso di nuovi attori tecnici, quali lo Smart Contract e l'Oracolo IoT, permette di soddisfare il softgoal di integrità grazie alla risorsa "registro immutabile". Gli attori umani smettono di dipendere dalla fiducia reciproca e iniziano a dipendere dal sistema per la validazione oggettiva dei dati (Figura \ref{fig:sd_to_be}).

\begin{figure}[ht!]
    \centering
    \includegraphics[width=\linewidth, keepaspectratio]{chapter1/image/sd_to_be.png}
    \caption{Modello SD: Supply Chain To-Be (Con Sistema Blockchain)}
    \label{fig:sd_to_be}
\end{figure}

Tuttavia, l'introduzione di tecnologia espone a nuove superfici d'attacco. Sono stati modellati tre profili di avversari: l'\textbf{Attaccante Interno} (es. operatore logistico corrotto), l'\textbf{Attaccante Esterno} (hacker remoto) e l'\textbf{Utente Maldestro} (operatore che commette errori non intenzionali). I diagrammi SR seguenti includono alberi di attacco che decompongono gli obiettivi malevoli in task operativi specifici.

\begin{figure}[ht!]
    \centering
    \includegraphics[width=\linewidth, keepaspectratio]{chapter1/image/sd_attackers.png}
    \caption{Modello SD: Sistema e Attaccanti con Alberi di Attacco}
    \label{fig:sd_attackers}
\end{figure}

\newpage
\begin{landscape}
    \begin{figure}[p]
        \centering
        \includegraphics[width=\linewidth, height=0.9\textheight, keepaspectratio]{chapter1/image/sr_attackers.png}
        \caption{Modello SR: Analisi degli Attaccanti}
        \label{fig:sr_attackers}
    \end{figure}
\end{landscape}

\section{Analisi delle Minacce DUAL-STRIDE}
L'analisi è stata condotta raggruppando gli asset secondo il paradigma DUAL-STRIDE, che estende il classico STRIDE (focalizzato sulla sicurezza) includendo aspetti di affidabilità e resilienza (DUAL).

Prima di esaminare le minacce, è essenziale definire gli asset critici del sistema che necessitano di protezione. La seguente tabella elenca le componenti primarie, spaziando dalla logica on-chain ai dispositivi fisici.

\newcolumntype{L}{>{\raggedright\arraybackslash}X}
\begin{tabularx}{\textwidth}{l l L l}
\toprule
\textbf{ID} & \textbf{Asset} & \textbf{Descrizione} & \textbf{Criticità} \\
\midrule
A1 & Smart Contract & logica di business e fondi in escrow & \textbf{Critica} \\
A2 & Evidenze IoT & dati dai sensori (E1-E5) & \textbf{Critica} \\
A3 & Pagamenti ETH & fondi depositati dai mittenti & \textbf{Critica} \\
A4 & Ruoli e Permessi & sistema AccessControl & \textbf{Alta} \\
A5 & CPT e Probabilità & parametri della rete bayesiana & \textbf{Alta} \\
A6 & Dati spedizioni & record on-chain e storico & \textbf{Media} \\
A7 & Interfaccia Web & frontend utente (DApp) & \textbf{Media} \\
A8 & Chiavi private & credenziali MetaMask & \textbf{Critica} \\
\bottomrule
\end{tabularx}

\vspace{1em}
La tabella successiva offre una visione d'insieme delle minacce identificate per ciascun asset, mappando le categorie STRIDE/DUAL sulle contromisure implementate.

% --- Colori Personalizzati Tabella Grande ---
\definecolor{headerblue}{RGB}{51, 122, 183}
\definecolor{stripeblue}{RGB}{235, 245, 255}
\definecolor{letterred}{RGB}{255, 0, 0}
\newcommand{\rotheader}[2]{%
    \rotatebox{90}{\parbox{3cm}{\raggedright \textcolor{letterred}{\textbf{#1}}#2}}}
\newcommand{\chk}{$\bullet$}

% --- INIZIO TABELLA GRANDE (LANDSCAPE) ---
\newgeometry{top=1.5cm, bottom=1.5cm, left=1.5cm, right=1.5cm}
\begin{landscape}

\subsection*{Riepilogo Visuale Minacce (DUAL-STRIDE)}
\setlength\tabcolsep{3pt}
\small
\begin{longtable}{|l|l|c|c|c|c|c|c|c|c|c|c|p{6cm}|p{5cm}|}
\hline
\rowcolor{headerblue}
\multicolumn{1}{|c|}{\textcolor{white}{\textbf{Asset}}} & 
\multicolumn{1}{c|}{\textcolor{white}{\textbf{Valore}}} & 
\rotheader{S}{poofing} & 
\rotheader{T}{ampering} & 
\rotheader{R}{epudiation} & 
\rotheader{I}{nformation disclosure} & 
\rotheader{D}{enaial of service} & 
\rotheader{E}{levation of privilege} & 
\rotheader{D}{anger} & 
\rotheader{U}{nreliability} & 
\rotheader{A}{bsence of resilience} & 
\rotheader{L}{eakage} & 
\multicolumn{1}{c|}{\textcolor{white}{\textbf{Descrizione Attacco}}} & 
\multicolumn{1}{c|}{\textcolor{white}{\textbf{Mitigazione (Control)}}} \\
\hline
\endfirsthead

\hline
\rowcolor{headerblue}
\multicolumn{1}{|c|}{\textcolor{white}{\textbf{Asset}}} & 
\multicolumn{1}{c|}{\textcolor{white}{\textbf{Valore}}} & 
\rotheader{S}{poofing} & 
\rotheader{T}{ampering} & 
\rotheader{R}{epudiation} & 
\rotheader{I}{nformation Disclosure} & 
\rotheader{D}{oS} & 
\rotheader{E}{levation of privilege} & 
\rotheader{D}{anger} & 
\rotheader{U}{nreliability} & 
\rotheader{A}{bsence of resilience} & 
\rotheader{L}{eakage/Exp.} & 
\multicolumn{1}{c|}{\textcolor{white}{\textbf{Descrizione Attacco}}} & 
\multicolumn{1}{c|}{\textcolor{white}{\textbf{Mitigazione (Control)}}} \\
\hline
\endhead

% A1
\rowcolor{stripeblue}
A1 & Critico & & \chk & & & & & \chk & & & & T1.1: manipolazione logica bytecode & Audit + Verifica formale \\
\hline
A1 & Critico & & & & & \chk & & & & \chk & & D1.1: blocco contratto (storage bomb) & Limiti Gas \\
\hline
\rowcolor{stripeblue}
A1 & Critico & & & & & & \chk & & & & \chk & E1.1: escalation privilegi admin & Est. Gnosis Safe Multi-sig \\
\hline

% A2
A2 & Critico & \chk & & & & & & \chk & & & & S2.1: impersonificazione sensore & Controllo Accessi Ruoli \\
\hline
\rowcolor{stripeblue}
A2 & Critico & & \chk & & & & & \chk & & & & T2.1: modifica evidenze in transito & Firma tx (Autenticazione Mittente) \\
\hline
A2 & Critico & & & \chk & & & & & & & & R2.1: ripudio invio evidenza & Logging Eventi \\
\hline
\rowcolor{stripeblue}
A2 & Critico & & & & \chk & & & & \chk & & & I2.1: corruzione dati sensore & Ridondanza E1-E5 \\
\hline

% A3
A3 & Critico & & \chk & & & & & \chk & & & & T3.1: attacco reentrancy & ReentrancyGuard \\
\hline
\rowcolor{stripeblue}
A3 & Critico & & & & & \chk & & & & \chk & & D3.1: blocco fondi escrow & Logica Rimborsi + Timeout \\
\hline
A3 & Critico & & & & & & \chk & & & & \chk & E3.1: drenaggio fondi contratto & Pagamento Push + ReentrancyGuard \\
\hline

% A4
\rowcolor{stripeblue}
A4 & Alto & \chk & & & & & & \chk & & & & S4.1: assegnazione ruolo fraudolenta & Concessione Solo Admin \\
\hline
A4 & Alto & & \chk & & & & & \chk & & & & T4.1: modifica mapping ruoli & Variabili Stato Private \\
\hline
\rowcolor{stripeblue}
A4 & Alto & & & & & & \chk & & & & \chk & E4.1: escalation privilegi & Controlli Multilivello \\
\hline

% A5
A5 & Alto & & \chk & & & & & \chk & & & & T5.1: manipolazione CPT & Variabili Private + Admin \\
\hline
\rowcolor{stripeblue}
A5 & Alto & & & & \chk & & & & & & \chk & I5.1: reverse engineering CPT & Offuscamento \\
\hline
A5 & Alto & & & & & & \chk & & & & \chk & E5.1: leak parametri bayesiani & Getter Privati \\
\hline

% A6
\rowcolor{stripeblue}
A6 & Medio & & & & \chk & & & & & & \chk & I6.1: data mining competitivo & Hashing dati sensibili \\
\hline
A6 & Medio & & \chk & & & & & & & & & T6.1: modifica storico spedizioni & Immutabilità blockchain \\
\hline
\rowcolor{stripeblue}
A6 & Medio & & & \chk & & & & & & & & R6.1: ripudio transazione & Log Eventi Permanenti \\
\hline

% A7
A7 & Medio & \chk & & & & & & \chk & & & & S7.1: phishing/Spoofing UI & Formazione Utente + HTTPS \\
\hline
\rowcolor{stripeblue}
A7 & Medio & & \chk & & & & & \chk & & & & T7.1: MITM su connessione Web3 & TLS + SRI \\
\hline
A7 & Medio & & & & \chk & & & & & & \chk & I7.1: XSS injection & Sanitizzazione Input \\
\hline

% A8
\rowcolor{stripeblue}
A8 & Critico & \chk & & & & & & \chk & & & & S8.1: furto chiavi (keylogger) & Hardware wallet \\
\hline
A8 & Critico & & & & \chk & & & & & & \chk & I8.1: esposizione seed phrase & Formazione Storage Sicuro \\
\hline
\rowcolor{stripeblue}
A8 & Critico & & & & & & \chk & & & & \chk & E8.1: accesso non autorizzato wallet & Policy Password + 2FA \\
\hline

\end{longtable}
\end{landscape}
\restoregeometry
\newpage

\section{Analisi Dettagliata degli Scenari (Abuse e Misuse Cases)}
A completamento dell'analisi tabellare, approfondiamo gli scenari più significativi distinguendo tra \textbf{Abuse Cases} (azioni malevole intenzionali) e \textbf{Misuse Cases} (errori non intenzionali o guasti), organizzati per area tematica.

% --- Colori Personalizzati Capitolo 3 (Abuse/Misuse) ---
\definecolor{abuseHeader}{RGB}{255, 140, 0}   % Arancione scuro
\definecolor{abuseCol}{RGB}{255, 229, 204}    % Arancione molto chiaro
\definecolor{misuseHeader}{RGB}{65, 105, 225} % Royal Blue
\definecolor{misuseCol}{RGB}{230, 240, 255}   % Blu molto chiaro

\subsection{Integrità dei Dati IoT (Asset A2)}
La validità del sistema dipende interamente dalla veridicità dei dati trasmessi dai sensori. Qui analizziamo il rischio di spoofing da parte di un attaccante esterno e l'inaffidabilità dovuta a guasti hardware.

\subsubsection*{ABUSE CASE: S2.1-A}
\begin{longtable}{| >{\columncolor{abuseCol}}p{0.25\textwidth} | p{0.67\textwidth} |}
\hline
\rowcolor{abuseHeader} \textcolor{white}{\textbf{Campo}} & \textcolor{white}{\textbf{Contenuto}} \\
\hline
\textbf{Case Type} & Abuse Case \\
\hline
\textbf{Use Case} & invio Evidenza Sensore \\
\hline
\textbf{Case ID} & S2.1-A \\
\hline
\textbf{Case Name} & \textbf{sensore malevolo falsificato} \\
\hline
\textbf{Actors} & attaccante esterno con chiave compromessa RUOLO\_SENSORE \\
\hline
\textbf{Description} & Un attaccante riesce ad ottenere la chiave privata di un account con RUOLO\_SENSORE (ad esempio tramite phishing, malware IoT o compromissione fisica del dispositivo) e la sfrutta per inviare evidenze falsificate allo smart contract, facendo sì che spedizioni non conformi vengano erroneamente validate. \\
\hline
\textbf{Data} & chiave privata RUOLO\_SENSORE, ID spedizione target, valori evidenze falsificati (E1-E5) \\
\hline
\textbf{Stimulus/Precond.} & - Spedizione in stato InAttesa \newline - Attaccante possiede chiave con RUOLO\_SENSORE \newline - Target: spedizione con merce deteriorata (temperatura fuori range) \\
\hline
\textbf{Basic Flow} & 1. L'attaccante individua la spedizione target monitorando gli eventi sulla blockchain \newline 2. Chiama \texttt{inviaEvidenza(idSpedizione, 1, true)} inserendo valori positivi \newline 3. Ripete l'operazione per E2-E5 con valori falsificati \newline 4. Il corriere chiama \texttt{validaEPaga()} e il sistema valida la spedizione \newline 5. Il corriere viene pagato per merce non conforme, a danno del mittente \\
\hline
\textbf{Alternative Flow} & - Se la transazione viene rifiutata per mancanza del ruolo, l'attacco fallisce senza conseguenze \newline - Se l'evidenza per quel sensore è già registrata, l'attaccante tenta con un altro account compromesso \\
\hline
\textbf{Exception Flow} & - Il sistema rileva che le evidenze arrivano troppo ravvicinate nel tempo (anomalia temporale) \newline - Invii multipli dallo stesso sensore per la stessa spedizione vengono bloccati automaticamente \\
\hline
\textbf{Response/Post.} & \textbf{Se l'attacco riesce:} il mittente paga per merce non conforme, che viene consegnata senza verifiche reali \newline \textbf{Tracciabilità:} tutte le transazioni restano registrate sulla blockchain, utili per analisi a posteriori \\
\hline
\textbf{Comments} & \textbf{CAPEC-151:} Identity Spoofing \newline \textbf{Contromisure consigliate:} rotazione periodica delle chiavi dei sensori, vincolo hardware tra chiave e dispositivo fisico, autenticazione tramite challenge-response \\
\hline
\end{longtable}

\subsubsection*{MISUSE CASE: S2.1-M}
\begin{longtable}{| >{\columncolor{misuseCol}}p{0.25\textwidth} | p{0.70\textwidth} |}
\hline
\rowcolor{misuseHeader} \textcolor{white}{\textbf{Campo}} & \textcolor{white}{\textbf{Contenuto}} \\
\hline
\textbf{Case Type} & Misuse Case \\
\hline
\textbf{Use Case} & invio Evidenza Sensore \\
\hline
\textbf{Case ID} & S2.1-M \\
\hline
\textbf{Case Name} & \textbf{guasto hardware sensore (Unreliability)} \\
\hline
\textbf{Actors} & sensore IoT difettoso/starato \\
\hline
\textbf{Description} & Un sensore IoT legittimamente autorizzato trasmette dati errati a causa di problemi hardware: deriva della calibrazione, batteria in esaurimento o interferenze elettromagnetiche nell'ambiente di trasporto. Questo può generare falsi positivi o negativi non intenzionali nella valutazione della spedizione. \\
\hline
\textbf{Data} & letture sensore errate, timestamp \\
\hline
\textbf{Stimulus/Precond.} & - Sensore installato da >6 mesi senza calibrazione \newline - Ambiente con interferenze elettromagnetiche \newline - Batteria sotto 20\% \\
\hline
\textbf{Basic Flow} & 1. Il sensore di temperatura presenta una deriva di +2\textdegree C \newline 2. Una spedizione mantenuta a 4\textdegree C viene letta come 6\textdegree C (fuori dal range 2-8\textdegree C) \newline 3. Il sensore invia \texttt{E1=false} in modo erroneo \newline 4. La rete bayesiana calcola una probabilità bassa di conformità \newline 5. La validazione fallisce e il mittente perde il pagamento depositato \\
\hline
\textbf{Alternative Flow} & - Gli altri sensori (E2-E5) compensano parzialmente l'errore di E1 grazie alla ridondanza bayesiana \newline - Se il timeout di 7 giorni scade prima della validazione, il mittente può richiedere il rimborso automatico \\
\hline
\textbf{Exception Flow} & - Il sistema di ridondanza rileva un'incongruenza tra i dati dei diversi sensori \newline - L'amministratore interviene manualmente per risolvere la situazione \\
\hline
\textbf{Response/Post.} & \textbf{Impatto:} falso negativo --- merce conforme viene rifiutata, il mittente subisce una perdita economica ingiusta \newline \textbf{Requisiti non funzionali:} contratti di manutenzione con SLA, calibrazione almeno semestrale, monitoraggio livello batteria \\
\hline
\textbf{Comments} & \textbf{DUA-Unreliability:} l'assenza di manutenzione preventiva è la causa principale \newline \textbf{Contromisure consigliate:} rilevamento automatico della deriva tramite algoritmi di apprendimento, consenso tra sensori multipli \\
\hline
\end{longtable}

\subsection{Sicurezza dei Pagamenti e Fondi (Asset A3)}
Gli smart contract gestiscono valore economico reale, rendendoli bersagli primari. Esaminiamo il classico attacco di Reentrancy e, parallelamente, la possibilità di errori umani o blocchi involontari dei fondi (DoS).

\subsubsection*{ABUSE CASE: T3.1-A}
\begin{longtable}{| >{\columncolor{abuseCol}}p{0.25\textwidth} | p{0.67\textwidth} |}
\hline
\rowcolor{abuseHeader} \textcolor{white}{\textbf{Campo}} & \textcolor{white}{\textbf{Contenuto}} \\
\hline
\textbf{Case Type} & Abuse Case \\
\hline
\textbf{Use Case} & pagamento Mittente \\
\hline
\textbf{Case ID} & T3.1-A \\
\hline
\textbf{Case Name} & \textbf{reentrancy attack su validaEPaga} \\
\hline
\textbf{Actors} & attaccante che deploya uno smart contract malevolo registrato come corriere \\
\hline
\textbf{Description} & L'attaccante deploya uno smart contract malevolo dotato di una funzione \texttt{receive()} che richiama automaticamente \texttt{validaEPaga()}. In assenza di protezioni, quando BNPagamenti esegue il trasferimento fondi al corriere tramite \texttt{.call\{value\}}, la fallback rientra nella funzione prima dell'aggiornamento dello stato, consentendo pagamenti multipli dalla stessa spedizione. \\
\hline
\textbf{Data} & contratto corriere malevolo con fallback, ID spedizione, saldo complessivo dello smart contract \\
\hline
\textbf{Stimulus/Precond.} & - Un mittente (complice o ignaro) crea una spedizione specificando il contratto malevolo come corriere \newline - La spedizione ha tutte le evidenze complete e valide \newline - Lo smart contract detiene fondi in escrow di più spedizioni attive \\
\hline
\textbf{Basic Flow} & 1. L'attaccante deploya un contratto con \texttt{receive()} malevola \newline 2. Una spedizione viene creata con questo contratto come corriere \newline 3. Le evidenze vengono inviate e risultano valide \newline 4. Il contratto malevolo (corriere) chiama \texttt{validaEPaga()} \newline 5. BNPagamenti esegue \texttt{corriere.call\{value: importo\}("")} \newline 6. La fallback ri-chiama \texttt{validaEPaga()}, \texttt{stato == InAttesa} ancora vero $\to$ secondo pagamento \newline 7. Il ciclo si ripete drenando i fondi nel contratto \\
\hline
\textbf{Alternative Flow} & - Se il ReentrancyGuard è attivo, la seconda chiamata viene immediatamente annullata con un revert \newline - Se si segue il pattern Checks-Effects-Interactions, lo stato viene aggiornato prima del trasferimento, impedendo la rientranza \\
\hline
\textbf{Exception Flow} & - Il ciclo si interrompe per esaurimento del gas \newline - Il saldo del contratto diventa insufficiente e la transazione va in revert \\
\hline
\textbf{Response/Post.} & \textbf{Se l'attacco riesce:} tutti i fondi presenti nel contratto vengono drenati \newline \textbf{Se l'attacco fallisce:} la transazione viene annullata, nessun pagamento avviene \\
\hline
\textbf{Comments} & \textbf{CAPEC-194:} Fake Resource Injection \newline \textbf{Contromisure implementate:} OpenZeppelin ReentrancyGuard, pattern CEI (lo stato viene aggiornato prima del trasferimento fondi), pagamento diretto protetto dal guard \\
\hline
\end{longtable}

\subsubsection*{MISUSE CASE: T3.1-M}
\begin{longtable}{| >{\columncolor{misuseCol}}p{0.25\textwidth} | p{0.70\textwidth} |}
\hline
\rowcolor{misuseHeader} \textcolor{white}{\textbf{Campo}} & \textcolor{white}{\textbf{Contenuto}} \\
\hline
\textbf{Case Type} & Misuse Case \\
\hline
\textbf{Use Case} & creazione Spedizione \\
\hline
\textbf{Case ID} & T3.1-M \\
\hline
\textbf{Case Name} & \textbf{errore indirizzo corriere (User Mistake)} \\
\hline
\textbf{Actors} & mittente distratto \\
\hline
\textbf{Description} & Il mittente inserisce per errore un indirizzo sbagliato nel campo ``Corriere'' durante la creazione della spedizione. Dato che la blockchain non verifica l'identità reale dietro un indirizzo, il pagamento finale verrà inviato all'indirizzo errato senza possibilità di annullamento. \\
\hline
\textbf{Data} & indirizzo corriere errato, fondi ETH depositati \\
\hline
\textbf{Stimulus/Precond.} & - Creazione spedizione tramite la DApp \newline - Inserimento manuale dell'indirizzo del corriere \newline - Nessuna validazione del checksum o dell'identità del destinatario \\
\hline
\textbf{Basic Flow} & 1. Il mittente inserisce l'indirizzo del corriere con un errore di battitura (ad esempio \texttt{0xAb...} invece di \texttt{0xAc...}) \newline 2. La spedizione viene completata regolarmente dai sensori \newline 3. La funzione \texttt{validaEPaga()} trasferisce i fondi all'indirizzo sbagliato \newline 4. Il corriere reale non riceve il compenso pattuito \\
\hline
\textbf{Alternative Flow} & - Se l'indirizzo errato ha un checksum valido, i fondi finiscono a uno sconosciuto o vengono ``bruciati'' \\
\hline
\textbf{Exception Flow} & - Un'interfaccia con rubrica indirizzi o lettura QR code previene l'errore di digitazione \\
\hline
\textbf{Response/Post.} & \textbf{Impatto:} perdita dei fondi versati, possibile contenzioso legale con il corriere reale \newline \textbf{Requisiti non funzionali:} interfaccia con selezione da elenco di corrieri certificati \\
\hline
\textbf{Comments} & \textbf{DUA-Exposure:} l'interfaccia espone l'utente a errori umani costosi \newline \textbf{Contromisure consigliate:} lista bianca di corrieri verificati, rubrica indirizzi integrata nella DApp \\
\hline
\end{longtable}

\subsubsection*{ABUSE CASE: D3.1-A}
\begin{longtable}{| >{\columncolor{abuseCol}}p{0.25\textwidth} | p{0.67\textwidth} |}
\hline
\rowcolor{abuseHeader} \textcolor{white}{\textbf{Campo}} & \textcolor{white}{\textbf{Contenuto}} \\
\hline
\textbf{Case Type} & Abuse Case \\
\hline
\textbf{Use Case} & validazione Spedizione \\
\hline
\textbf{Case ID} & D3.1-A \\
\hline
\textbf{Case Name} & \textbf{withholding attack (Denial of Service)} \\
\hline
\textbf{Actors} & sensore compromesso / Corriere collusivo \\
\hline
\textbf{Description} & Un attaccante che controlla un sensore (ad esempio E5) oppure un corriere collusivo rifiuta intenzionalmente di inviare l'ultima evidenza necessaria alla validazione, bloccando così i fondi del mittente in escrow come forma di estorsione o per arrecare danno reputazionale al sistema. \\
\hline
\textbf{Data} & ID spedizione, evidenze E1-E4 (inviate), E5 (trattenuta) \\
\hline
\textbf{Stimulus/Precond.} & - Spedizione in stato InAttesa \newline - Evidenze E1-E4 già registrate on-chain \newline - L'attaccante controlla il sensore E5 oppure l'account corriere \newline - Scenario ipotetico senza meccanismo di timeout \\
\hline
\textbf{Basic Flow} & 1. La spedizione procede normalmente nelle fasi iniziali \newline 2. I sensori E1-E4 inviano regolarmente le loro evidenze \newline 3. L'attaccante trattiene deliberatamente E5 a tempo indeterminato \newline 4. Senza tutte le evidenze, \texttt{validaEPaga()} non può essere chiamata \newline 5. I fondi del mittente restano bloccati nel contratto \\
\hline
\textbf{Alternative Flow} & - L'attaccante chiede un riscatto in cambio dell'invio di E5 \newline - Più sensori vengono compromessi contemporaneamente (E3+E4+E5) per rendere l'attacco più robusto \\
\hline
\textbf{Exception Flow} & - Allo scadere del timeout di 7 giorni, il mittente può chiamare \texttt{richiestaRimborso()} per recuperare i fondi \newline - L'amministratore interviene in caso di emergenza \\
\hline
\textbf{Response/Post.} & \textbf{Se l'attacco riesce:} blocco temporaneo dei fondi, danno alla reputazione del sistema, possibile estorsione \newline \textbf{Contromisura attiva:} rimborso automatico allo scadere del timeout \\
\hline
\textbf{Comments} & \textbf{CAPEC-227:} Sustained Client Engagement \newline \textbf{Contromisure:} timeout automatico con logica di rimborso dopo 3 tentativi di validazione falliti \\
\hline
\end{longtable}

\subsubsection*{MISUSE CASE: D3.1-M}
\begin{longtable}{| >{\columncolor{misuseCol}}p{0.25\textwidth} | p{0.70\textwidth} |}
\hline
\rowcolor{misuseHeader} \textcolor{white}{\textbf{Campo}} & \textcolor{white}{\textbf{Contenuto}} \\
\hline
\textbf{Case Type} & Misuse Case \\
\hline
\textbf{Use Case} & validazione Spedizione \\
\hline
\textbf{Case ID} & D3.1-M \\
\hline
\textbf{Case Name} & \textbf{timeout connettività IoT (Absence of Resilience)} \\
\hline
\textbf{Actors} & sensore offline per guasto rete/alimentazione \\
\hline
\textbf{Description} & Un sensore legittimo perde la connettività durante il trasporto (batteria esaurita, zona senza copertura di rete, guasto al modem) e non riesce a trasmettere le evidenze, causando un blocco involontario dei fondi del mittente. \\
\hline
\textbf{Data} & evidenze trasmesse solo parzialmente, log di connettività del sensore \\
\hline
\textbf{Stimulus/Precond.} & - Spedizione in transito attraverso una zona remota con scarsa copertura \newline - Livello batteria del sensore sotto il 10\% \newline - Nessun meccanismo di memorizzazione locale e invio differito \\
\hline
\textbf{Basic Flow} & 1. La spedizione parte con tutti i sensori funzionanti \newline 2. Attraversando una zona montuosa si perde il segnale GSM \newline 3. Le evidenze E1 ed E2 vengono inviate, ma E3-E5 non vengono mai trasmesse \newline 4. La batteria del sensore si esaurisce \newline 5. La spedizione arriva a destinazione, ma senza evidenze complete \newline 6. I fondi del mittente restano bloccati (non intenzionalmente) \\
\hline
\textbf{Alternative Flow} & - Il sensore recupera la connessione appena in tempo e riesce a trasmettere le evidenze mancanti \newline - Le evidenze vengono recuperate manualmente dal logger interno del dispositivo \\
\hline
\textbf{Exception Flow} & - Il sistema rileva il timeout di 7 giorni e attiva il rimborso automatico al mittente \newline - Il corriere fornisce documentazione alternativa delle evidenze mancanti \\
\hline
\textbf{Response/Post.} & \textbf{Impatto:} il mittente deve attendere 7 giorni per il rimborso, il servizio risulta degradato, costi di opportunità \newline \textbf{Requisiti non funzionali:} SLA con garanzia di disponibilità al 99.5\%, batterie sostituibili a caldo, doppia SIM con failover automatico \\
\hline
\end{longtable}

\subsection{Configurazione e Logica Bayesiana (Asset A5 e A6)}
Questi scenari si concentrano sulla manipolazione della logica decisionale del sistema (le CPT) e sul furto di informazioni sensibili tramite reverse engineering o data mining.

\subsubsection*{ABUSE CASE: T5.1-A}
\begin{longtable}{| >{\columncolor{abuseCol}}p{0.25\textwidth} | p{0.67\textwidth} |}
\hline
\rowcolor{abuseHeader} \textcolor{white}{\textbf{Campo}} & \textcolor{white}{\textbf{Contenuto}} \\
\hline
\textbf{Case Type} & Abuse Case \\
\hline
\textbf{Use Case} & configurazione Rete Bayesiana \\
\hline
\textbf{Case ID} & T5.1-A \\
\hline
\textbf{Case Name} & \textbf{insider admin attack su parametri bayesiani} \\
\hline
\textbf{Actors} & admin infedele con RUOLO\_ORACOLO \\
\hline
\textbf{Description} & Un amministratore con privilegi RUOLO\_ORACOLO modifica intenzionalmente le tabelle di probabilità condizionate (CPT) per alterare la logica decisionale del sistema, favorendo validazioni fraudolente --- ad esempio impostando P(Esito$|$Freschezza=false) al 95\% invece del corretto 10\%. \\
\hline
\textbf{Data} & variabili CPT (cptEsitoFrFr, cptEsitoFrNF, cptFranchezzaAmb), chiave dell'amministratore \\
\hline
\textbf{Stimulus/Precond.} & - Admin con RUOLO\_ORACOLO compromesso (corruzione, ricatto, insider threat) \newline - Parametri CPT modificabili tramite la funzione \texttt{impostaCPT()} \newline - Evento \texttt{CPTImpostata} registra le modifiche, ma in assenza di monitoraggio attivo nessuno le nota \\
\hline
\textbf{Basic Flow} & 1. L'admin chiama \texttt{impostaCPT()} con valori fraudolenti \newline 2. Imposta \texttt{P(Esito|Freschezza=false, \allowbreak Franchezza=false) = 99\%} (dovrebbe essere $\sim$5\%) \newline 3. Qualsiasi spedizione, anche non conforme, supera la validazione \newline 4. Mittenti collusivi ricevono pagamenti indebiti \newline 5. L'integrità del sistema di calcolo risulta completamente compromessa \\
\hline
\textbf{Alternative Flow} & - L'admin riduce le soglie (ad esempio \texttt{P(Esito|true,true) = 1\%}) per negare pagamenti a spedizioni legittime \\
\hline
\textbf{Exception Flow} & - Il monitoraggio rileva un tasso di approvazione anomalo (99\%) e genera un allarme \newline - Una governance multi-firma richiede l'approvazione di almeno 2 su 3 amministratori per modificare le CPT \\
\hline
\textbf{Response/Post.} & \textbf{Se l'attacco riesce:} perdita di integrità del sistema, validazioni prive di significato, danno finanziario diffuso \newline \textbf{Rilevamento:} analisi statistica dei tassi di validazione, alert automatici sulle anomalie \\
\hline
\textbf{Comments} & \textbf{CAPEC-122:} Privilege Abuse \newline \textbf{Contromisure consigliate:} governance multi-firma, parametri CPT bloccati dopo il deployment, votazione tramite DAO \\
\hline
\end{longtable}

\subsubsection*{MISUSE CASE: T5.1-M}
\begin{longtable}{| >{\columncolor{misuseCol}}p{0.25\textwidth} | p{0.70\textwidth} |}
\hline
\rowcolor{misuseHeader} \textcolor{white}{\textbf{Campo}} & \textcolor{white}{\textbf{Contenuto}} \\
\hline
\textbf{Case Type} & Misuse Case \\
\hline
\textbf{Use Case} & configurazione Rete Bayesiana \\
\hline
\textbf{Case ID} & T5.1-M \\
\hline
\textbf{Case Name} & \textbf{errore configurazione probabilità (Human Error)} \\
\hline
\textbf{Actors} & admin inesperto/distratto \\
\hline
\textbf{Description} & Un amministratore commette un errore non intenzionale durante la configurazione delle CPT: un errore di battitura (scrive 150 anziché 15), un'inversione della condizionalità probabilistica, o un valore fuori scala. \\
\hline
\textbf{Data} & parametri CPT errati, hash della transazione di configurazione \\
\hline
\textbf{Stimulus/Precond.} & - L'admin aggiorna le CPT per incorporare un nuovo modello bayesiano \newline - La validazione del range 0-100 è implementata, ma non copre errori di tipo semantico \newline - Non esiste un ambiente di staging per effettuare test prima della messa in produzione \\
\hline
\textbf{Basic Flow} & 1. L'admin intende impostare un valore CPT pari a 15 (15\%) \newline 2. Digita erroneamente 150 --- ma il \texttt{require(\_cpt <= 100)} lo blocca con un revert \newline 3. Se però non fosse presente la validazione, il valore verrebbe accettato producendo risultati incoerenti \newline 4. I calcoli probabilistici restituirebbero risultati fuori scala \newline 5. Le validazioni diventerebbero casuali e non basate sulle evidenze reali \\
\hline
\textbf{Alternative Flow} & - Errore di scala decimale: l'admin scrive \texttt{0.85} come \texttt{85} (il sistema lo interpreta come 8500\%) \newline - L'admin inverte la probabilità condizionata: inserisce P(Esito$|$Freschezza) al posto di P(Freschezza$|$Esito) \\
\hline
\textbf{Exception Flow} & - Il controllo \texttt{require(value <= 100)} impedisce i valori fuori range \newline - Un test in ambiente di staging rileva l'incongruenza prima della messa in produzione \\
\hline
\textbf{Response/Post.} & \textbf{Impatto:} sistema instabile con decisioni errate fino al rollback, validazioni e rimborsi potenzialmente scorretti \newline \textbf{Requisiti non funzionali:} validazione degli input, test unitari sulle configurazioni, rilascio graduale \\
\hline
\end{longtable}

\subsubsection*{ABUSE CASE: I6.1-A}
\begin{longtable}{| >{\columncolor{abuseCol}}p{0.25\textwidth} | p{0.67\textwidth} |}
\hline
\rowcolor{abuseHeader} \textcolor{white}{\textbf{Campo}} & \textcolor{white}{\textbf{Contenuto}} \\
\hline
\textbf{Case Type} & Abuse Case \\
\hline
\textbf{Use Case} & consultazione Storico Spedizioni \\
\hline
\textbf{Case ID} & I6.1-A \\
\hline
\textbf{Case Name} & \textbf{analisi competitiva blockchain} \\
\hline
\textbf{Actors} & concorrente commerciale, data analyst \\
\hline
\textbf{Description} & Un'azienda concorrente analizza le transazioni pubbliche registrate sulla blockchain per estrarre informazioni di intelligence competitiva: volumi di spedizioni, rotte ricorrenti, clienti abituali, periodi di picco e stime dei margini operativi (desumibili dai valori ETH). \\
\hline
\textbf{Data} & eventi \texttt{SpedizioneCreata}, \texttt{SpedizionePagata}, indirizzi mittenti/corrieri, importi ETH \\
\hline
\textbf{Stimulus/Precond.} & - Blockchain pubblica (Besu in modalità permissioned, ma i log degli eventi restano accessibili) \newline - Gli eventi emessi non sono offuscati \newline - Gli indirizzi possono essere correlati a identità reali (fuga di dati KYC, ingegneria sociale) \\
\hline
\textbf{Basic Flow} & 1. Il concorrente raccoglie tutti gli eventi \texttt{SpedizioneCreata} a partire dal blocco iniziale \newline 2. Raggruppa le transazioni per indirizzo del mittente e identifica i clienti principali \newline 3. Analizza i timestamp per rilevare picchi stagionali (ad esempio campagne vaccinali) \newline 4. Correla gli importi in ETH per stimare volumi e margini operativi \newline 5. Utilizza queste informazioni per proporre prezzi più bassi o acquisire clienti chiave \\
\hline
\textbf{Alternative Flow} & - Analisi dei pattern delle rotte per ottimizzazione logistica a proprio vantaggio \newline - Identificazione delle validazioni fallite per individuare debolezze del concorrente \\
\hline
\textbf{Exception Flow} & - L'hashing dei parametri sensibili (destinazioni, prodotti) limita la fuga di informazioni \newline - Prove a conoscenza zero sugli importi nascondono i margini operativi \\
\hline
\textbf{Response/Post.} & \textbf{Impatto:} perdita del vantaggio competitivo, esposizione di informazioni commerciali riservate \newline \textbf{Aspetto legale:} potenziale violazione del segreto industriale (a seconda della giurisdizione) \\
\hline
\textbf{Comments} & \textbf{CAPEC-116:} Excavation (raccolta informazioni) \newline \textbf{Contromisure consigliate:} hash dei dati sensibili, accesso in lettura riservato agli stakeholder, transazioni private (Aztec, zkSNARK) \\
\hline
\end{longtable}

\subsubsection*{ABUSE CASE: I5.1-A}
\begin{longtable}{| >{\columncolor{abuseCol}}p{0.25\textwidth} | p{0.67\textwidth} |}
\hline
\rowcolor{abuseHeader} \textcolor{white}{\textbf{Campo}} & \textcolor{white}{\textbf{Contenuto}} \\
\hline
\textbf{Case Type} & Abuse Case \\
\hline
\textbf{Use Case} & lettura Parametri Bayesiani \\
\hline
\textbf{Case ID} & I5.1-A \\
\hline
\textbf{Case Name} & \textbf{storage slot reading via Web3} \\
\hline
\textbf{Actors} & attaccante/competitor \\
\hline
\textbf{Description} & Anche se le variabili CPT sono dichiarate \texttt{private}, un attaccante può usare \texttt{eth\_getStorageAt} per leggere direttamente gli slot di storage del contratto e decompilare il bytecode per ricostruire la mappatura dei parametri della rete bayesiana. \\
\hline
\textbf{Data} & bytecode del contratto, layout dello storage, valori delle CPT \\
\hline
\textbf{Stimulus/Precond.} & - Contratto pubblicato su una blockchain pubblica \newline - Le variabili CPT sono dichiarate \texttt{private} ma restano comunque leggibili via RPC \newline - L'attaccante conosce il layout dello storage di Solidity \\
\hline
\textbf{Basic Flow} & 1. L'attaccante ottiene l'indirizzo del contratto \newline 2. Usa \texttt{web3.eth.getStorageAt(address, slot)} per ogni slot da 0 a 20 \newline 3. Identifica il pattern delle strutture CPT nello storage \newline 4. Decompila il bytecode con strumenti specializzati (Dedaub, Etherscan) \newline 5. Ricostruisce la rete bayesiana completa \newline 6. Crea un sistema concorrente con la stessa logica (furto di proprietà intellettuale) \\
\hline
\textbf{Response/Post.} & \textbf{Impatto:} furto del modello bayesiano come proprietà intellettuale, replicazione del sistema, perdita del vantaggio competitivo \\
\hline
\textbf{Comments} & \textbf{CAPEC-188:} Reverse Engineering \newline \textbf{Contromisure consigliate:} calcolo off-chain con oracolo fidato (Chainlink Functions), prove zkSNARK per validare senza rivelare i parametri \\
\hline
\end{longtable}

\subsection{Interfaccia Utente e Credenziali (Asset A7 e A8)}
Infine, analizziamo il vettore di attacco più comune: il fattore umano. Dagli attacchi di phishing (spoofing UI) al furto o smarrimento delle chiavi private, questi scenari eludono spesso le difese crittografiche del contratto.

\subsubsection*{ABUSE CASE: S7.1-A}
\begin{longtable}{| >{\columncolor{abuseCol}}p{0.25\textwidth} | p{0.67\textwidth} |}
\hline
\rowcolor{abuseHeader} \textcolor{white}{\textbf{Campo}} & \textcolor{white}{\textbf{Contenuto}} \\
\hline
\textbf{Case Type} & Abuse Case \\
\hline
\textbf{Use Case} & accesso DApp \\
\hline
\textbf{Case ID} & S7.1-A \\
\hline
\textbf{Case Name} & \textbf{cloning DApp phishing} \\
\hline
\textbf{Actors} & attaccante phisher, utente vittima \\
\hline
\textbf{Description} & L'attaccante crea una replica identica della DApp legittima (un clone del frontend) e la ospita su un dominio dal nome simile (typosquatting: ad esempio \texttt{bayesian-oracIe.com} con la I maiuscola al posto della L). L'obiettivo è sottrarre la seed phrase dell'utente o fargli firmare transazioni malevole. \\
\hline
\textbf{Data} & seed phrase di MetaMask, chiavi private, token di approvazione \\
\hline
\textbf{Stimulus/Precond.} & - L'utente riceve un link di phishing via email o social network \newline - Il frontend della DApp non è verificato (nessun badge ENS) \newline - L'utente non controlla l'URL nella barra degli indirizzi \\
\hline
\textbf{Basic Flow} & 1. L'attaccante registra il dominio \texttt{bayesian-0racle.com} (con lo zero al posto della O) \newline 2. Vi pubblica un clone della DApp con un backend malevolo \newline 3. Invia un'email: ``Aggiorna il wallet per la nuova versione'' \newline 4. L'utente clicca sul link e raggiunge il sito fasullo \newline 5. Compare un popup: ``Riconnetti MetaMask'' $\to$ l'utente inserisce la seed phrase \newline 6. L'attaccante cattura la seed e svuota il wallet \\
\hline
\textbf{Alternative Flow} & - Richiesta fasulla di ``approvazione contratto'' $\to$ l'utente firma \texttt{approve(attackerAddress, MAX\_UINT)} \newline - Iniezione di script malevolo tramite una CDN compromessa \\
\hline
\textbf{Response/Post.} & \textbf{Se l'attacco riesce:} furto completo dei fondi nel wallet, compromissione dell'identità on-chain \newline \textbf{Requisiti non funzionali:} formazione sulla sicurezza, HTTPS obbligatorio, header CSP \\
\hline
\textbf{Comments} & \textbf{CAPEC-98:} Phishing \newline \textbf{Contromisure consigliate:} HTTPS con HSTS, verifica del nome ENS, collegamento sicuro tramite WalletConnect, avvisi educativi integrati nell'interfaccia \\
\hline
\end{longtable}

\subsubsection*{MISUSE CASE: S7.1-M}
\begin{longtable}{| >{\columncolor{misuseCol}}p{0.25\textwidth} | p{0.70\textwidth} |}
\hline
\rowcolor{misuseHeader} \textcolor{white}{\textbf{Campo}} & \textcolor{white}{\textbf{Contenuto}} \\
\hline
\textbf{Case Type} & Misuse Case \\
\hline
\textbf{Use Case} & creazione Spedizione \\
\hline
\textbf{Case ID} & S7.1-M \\
\hline
\textbf{Case Name} & \textbf{errore input parametri (User Mistake)} \\
\hline
\textbf{Actors} & mittente distratto \\
\hline
\textbf{Description} & Un utente legittimo interagisce con la DApp autentica, ma commette un errore durante l'inserimento dei dati: indirizzo del corriere sbagliato, importo in ETH errato (1 ETH al posto di 0.1), parametro della spedizione invertito. \\
\hline
\textbf{Data} & parametri della transazione errati, commissioni gas \\
\hline
\textbf{Stimulus/Precond.} & - Form di creazione spedizione con campi a inserimento libero \newline - Validazione client-side presente ma non esaustiva su tutti i campi \newline - L'utente commette un errore durante il copia-incolla \\
\hline
\textbf{Basic Flow} & 1. Il mittente crea una spedizione con \texttt{valore = 1 ETH} (voleva scrivere 0.1) \newline 2. Conferma la transazione su MetaMask senza rileggere i dettagli \newline 3. La transazione viene eseguita con un importo 10 volte superiore \newline 4. La validazione ha esito positivo $\to$ il corriere riceve 1 ETH invece di 0.1 \newline 5. Il mittente perde 0.9 ETH senza possibilità di annullamento \\
\hline
\textbf{Response/Post.} & \textbf{Impatto:} perdita economica per l'utente, impossibile annullare la transazione una volta confermata \newline \textbf{Requisiti non funzionali:} validazione nell'interfaccia (menù a tendina, slider, limiti massimi), finestra di conferma dettagliata \\
\hline
\textbf{Comments} & \textbf{DUA-Exposure:} un'interfaccia non a prova di errore espone gli utenti a sbagli costosi \newline \textbf{Contromisure consigliate:} vincoli sugli input, anteprima della transazione prima della firma, simulazione con periodo di attesa \\
\hline
\end{longtable}

\subsubsection*{ABUSE CASE: S8.1-A}
\begin{longtable}{| >{\columncolor{abuseCol}}p{0.25\textwidth} | p{0.67\textwidth} |}
\hline
\rowcolor{abuseHeader} \textcolor{white}{\textbf{Campo}} & \textcolor{white}{\textbf{Contenuto}} \\
\hline
\textbf{Case Type} & Abuse Case \\
\hline
\textbf{Use Case} & gestione Wallet \\
\hline
\textbf{Case ID} & S8.1-A \\
\hline
\textbf{Case Name} & \textbf{keylogger/malware exfiltration} \\
\hline
\textbf{Actors} & attaccante con malware installato, vittima \\
\hline
\textbf{Description} & L'attaccante installa un keylogger o un dirottatore degli appunti sul dispositivo della vittima per catturare la seed phrase durante il ripristino del wallet, oppure estrae le chiavi da uno storage non cifrato. \\
\hline
\textbf{Data} & seed phrase (12 o 24 parole), file keystore JSON, password di MetaMask \\
\hline
\textbf{Stimulus/Precond.} & - L'utente installa software compromesso (falso airdrop, software piratato) \newline - Il keylogger è attivo in background \newline - L'utente apre MetaMask per ripristinare il wallet \\
\hline
\textbf{Basic Flow} & 1. L'utente scarica un ``wallet optimizer tool'' malevolo \newline 2. Il malware installa un keylogger e un monitor degli appunti \newline 3. L'utente avvia il ripristino di MetaMask e digita la seed phrase \newline 4. Il keylogger cattura le parole e le invia al server di comando \newline 5. L'attaccante importa la seed nel proprio wallet \newline 6. Svuota immediatamente tutti i fondi (ETH e token) \\
\hline
\textbf{Response/Post.} & \textbf{Se l'attacco riesce:} furto totale dei fondi, compromissione permanente dell'identità (la seed non può essere cambiata) \newline \textbf{Rilevamento:} lo svuotamento rapido del wallet è un segnale d'allarme, ma solitamente è già troppo tardi \\
\hline
\textbf{Comments} & \textbf{ATT\&CK-T1056.001:} Keylogging \newline \textbf{CAPEC-568:} Capture Credentials via Keylogger \newline \textbf{Contromisure consigliate:} hardware wallet obbligatorio per asset di valore elevato, irrobustimento del sistema operativo, antivirus aggiornato \\
\hline
\end{longtable}

\subsubsection*{MISUSE CASE: S8.1-M}
\begin{longtable}{| >{\columncolor{misuseCol}}p{0.25\textwidth} | p{0.70\textwidth} |}
\hline
\rowcolor{misuseHeader} \textcolor{white}{\textbf{Campo}} & \textcolor{white}{\textbf{Contenuto}} \\
\hline
\textbf{Case Type} & Misuse Case \\
\hline
\textbf{Use Case} & backup Wallet \\
\hline
\textbf{Case ID} & S8.1-M \\
\hline
\textbf{Case Name} & \textbf{seed phrase salvata in chiaro (User Negligence)} \\
\hline
\textbf{Actors} & utente inesperto/negligente \\
\hline
\textbf{Description} & L'utente salva la seed phrase in un formato non sicuro: screenshot sincronizzato nel cloud, email inviata a sé stesso, note sullo smartphone non cifrate, foto del backup fisico caricata su Google Photos. \\
\hline
\textbf{Data} & seed phrase in chiaro, backup nel cloud \\
\hline
\textbf{Stimulus/Precond.} & - Utente alle prime armi con le criptovalute, che non comprende la gravità della seed \newline - Backup automatico nel cloud attivo per impostazione predefinita (iCloud, Google Drive) \newline - Nessuno strumento di gestione dei segreti \\
\hline
\textbf{Basic Flow} & 1. Alla creazione del wallet, MetaMask genera la seed phrase \newline 2. L'utente fa uno screenshot su iPhone ``per sicurezza'' \newline 3. iPhone carica automaticamente la foto su iCloud (impostazione predefinita) \newline 4. L'account iCloud viene compromesso (phishing, password debole, assenza 2FA) \newline 5. L'attaccante accede a Foto di iCloud e trova lo screenshot della seed \newline 6. Importa il wallet e svuota i fondi \\
\hline
\textbf{Response/Post.} & \textbf{Impatto:} furto dei fondi per negligenza, nessuna possibilità di recupero \newline \textbf{Requisiti non funzionali:} tutorial educativo obbligatorio al primo avvio, avviso di MetaMask contro gli screenshot della seed \\
\hline
\textbf{Comments} & \textbf{DUA-Exposure:} mancanza di consapevolezza sulla sicurezza delle criptovalute \newline \textbf{Contromisure consigliate:} tutorial integrato sulla protezione della seed, blocco screenshot durante la visualizzazione, frazionamento tramite Shamir Secret Sharing (2 su 3 parti) \\
\hline
\end{longtable}

\subsubsection*{ABUSE CASE: E4.1-A}
\begin{longtable}{| >{\columncolor{abuseCol}}p{0.25\textwidth} | p{0.67\textwidth} |}
\hline
\rowcolor{abuseHeader} \textcolor{white}{\textbf{Campo}} & \textcolor{white}{\textbf{Contenuto}} \\
\hline
\textbf{Case Type} & Abuse Case \\
\hline
\textbf{Use Case} & assegnazione Ruoli \\
\hline
\textbf{Case ID} & E4.1-A \\
\hline
\textbf{Case Name} & \textbf{privilege escalation via function selector collision} \\
\hline
\textbf{Actors} & attaccante esperto Solidity \\
\hline
\textbf{Description} & L'attaccante sfrutta una (ipotetica) collisione sugli hash delle firme delle funzioni o una vulnerabilità legata a \texttt{delegatecall} per aggirare il modificatore \texttt{onlyRole} e auto-assegnarsi il RUOLO\_ORACOLO o il DEFAULT\_ADMIN\_ROLE. \\
\hline
\textbf{Data} & collisione sulla firma della funzione, payload delegatecall \\
\hline
\textbf{Stimulus/Precond.} & - Il contratto non usa delegatecall (scenario ipotetico) \newline - Collisione sugli hash dei selettori di funzione (birthday attack su 4 byte) \newline - Assenza di check strict sull'origine della chiamata \\
\hline
\textbf{Basic Flow} & 1. L'attaccante identifica il selettore della funzione \texttt{grantRole(bytes32,address)}: \texttt{0x2f2ff15d} \newline 2. Cerca una funzione apparentemente innocua con lo stesso hash troncato \newline 3. Costruisce un payload che aggira il modificatore tramite la collisione \newline 4. Chiama la funzione ``innocua'' che esegue internamente \texttt{grantRole} \newline 5. Si auto-assegna il DEFAULT\_ADMIN\_ROLE \newline 6. Assume il controllo completo del sistema \\
\hline
\textbf{Response/Post.} & \textbf{Se l'attacco riesce:} presa di controllo completa del contratto, possibilità di modificare la logica e drenare i fondi \newline \textbf{Gravità:} CRITICA \\
\hline
\textbf{Comments} & \textbf{CAPEC-122:} Privilege Abuse \newline \textbf{ATT\&CK-T1078:} Valid Accounts (escalation dei privilegi) \newline \textbf{Contromisure:} utilizzo dell'ultima versione di OpenZeppelin, evitare delegatecall con input utente, verifica formale del controllo degli accessi \\
\hline
\end{longtable}

\section{Standard Internazionali e Mitigazioni}
L'analisi non si è limitata all'identificazione dei rischi, ma ha provveduto a mapparli sui framework standard di settore per garantire una copertura esaustiva. La tabella seguente correla le minacce discusse con i pattern \textbf{CAPEC} (Common Attack Pattern Enumeration and Classification) e le tattiche \textbf{MITRE ATT\&CK}.

\begin{tabularx}{\textwidth}{l l l L l}
\toprule
\textbf{Threat ID} & \textbf{STRIDE} & \textbf{CAPEC ID} & \textbf{CAPEC Name} & \textbf{ATT\&CK TTP} \\
\midrule
S2.1 & Spoofing & CAPEC-151 & identity Spoofing & T1134 \\
T2.1 & Tampering & CAPEC-94 & man-in-the-Middle & T1557 \\
T3.1 & Tampering & CAPEC-194 & fake Resource Injection & - \\
T5.1 & Tampering & CAPEC-1 & accessing Functionality Not Properly Constrained & T1078 \\
D3.1 & Denial & CAPEC-227 & sustained Client Engagement & T1499 \\
I6.1 & Info Disclosure & CAPEC-116 & excavation & T1213 \\
I5.1 & Info Disclosure & CAPEC-188 & reverse Engineering & - \\
S7.1 & Spoofing & CAPEC-98 & phishing & T1566.002 \\
S8.1 & Spoofing & CAPEC-568 & capture Credentials via Keylogger & T1056.001 \\
E4.1 & Elevation & CAPEC-122 & privilege Abuse & T1078.004 \\
\bottomrule
\end{tabularx}

\vspace{1em}
A fronte di queste minacce, sono state implementate contromisure specifiche (Mitigations) progettate per ridurre il rischio residuo a livelli accettabili:

\begin{enumerate}
    \item \textbf{Sicurezza degli Smart Contract (A1, A3, A4):}
    È stata adottata una strategia di difesa in profondità che include l'uso di librerie standard auditate (OpenZeppelin AccessControl e ReentrancyGuard), l'implementazione del pattern Checks-Effects-Interactions per prevenire la rientranza e meccanismi di "Emergency Stop" (Circuit Breaker) per congelare il sistema in caso di anomalie gravi.

    \item \textbf{Integrità dei Dati e IoT (A2, A5):}
    Per mitigare l'inaffidabilità dei sensori, il sistema non si fida del singolo dato ma utilizza la ridondanza bayesiana. Inoltre, è stato imposto un rate-limiting (1 minuto) per prevenire lo spam di dati e la validazione rigorosa degli input per i parametri CPT (range 0-100).

    \item \textbf{Protezione dell'Utente (A7, A8):}
    Dato che il fattore umano è l'anello debole non controllabile via codice, le mitigazioni si spostano sul piano educativo e dell'interfaccia: validazione client-side aggressiva, feedback visivi chiari prima della firma delle transazioni e la raccomandazione imperativa di utilizzare Hardware Wallet per la gestione delle chiavi private.
\end{enumerate}

\section{Conclusioni e Rischio Residuo}
L'analisi \textbf{DUAL-STRIDE} ha permesso di identificare 25 scenari di minaccia, dettagliando 9 abuse case e 6 misuse case.
La valutazione finale dei rischi residui mostra un quadro positivo per quanto riguarda la sicurezza intrinseca del codice, ma evidenzia la criticità permanente legata alla gestione delle credenziali utente.

\small
\begin{tabularx}{\textwidth}{l l X l X}
\toprule
\textbf{Asset} & \textbf{Rischio inerente} & \textbf{Mitigazioni} & \textbf{Rischio residuo} & \textbf{Accettazione} \\
\midrule
A1 & CRITICO & Audit + Verifica formale & BASSO & $\checkmark$ Accettato \\
A2 & CRITICO & HSM + Firme digitali & MEDIO & $\checkmark$ Accettato \\
A3 & CRITICO & ReentrancyGuard + Timeout & BASSO & $\checkmark$ Accettato \\
A4 & ALTO & AccessControl + Eventi & BASSO & $\checkmark$ Accettato \\
A5 & ALTO & Variabili Private + Validazione input & MEDIO & $\checkmark$ Accettato \\
A6 & MEDIO & Hashing & BASSO & $\checkmark$ Accettato \\
A7 & MEDIO & Validazione input & MEDIO & \textbf{!} Formazione necessaria \\
A8 & CRITICO & Raccomandazione HW Wallet & ALTO & \textbf{!} Responsabilità utente \\
\bottomrule
\end{tabularx}

In conclusione, mentre gli asset tecnici (A1-A6) sono stati messi in sicurezza con successo, l'asset \textbf{A8 (Chiavi Private)} rimane l'elemento più vulnerabile, non mitigabile tecnologicamente se non tramite l'educazione dell'utente finale.