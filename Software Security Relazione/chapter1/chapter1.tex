\chapter{Contesto e Obiettivi}

\begin{preamble}
In questo capitolo viene analizzato il dominio della Pharmaceutical Cold Chain, evidenziando le criticità di sicurezza attuali. Vengono quindi definiti gli obiettivi del progetto: automazione fidata, integrità dei dati e resilienza agli attacchi.
\end{preamble}

\section{Il problema della Catena del Freddo}
La spedizione di medicinali sensibili è un processo ad alto rischio. Secondo l'Organizzazione Mondiale della Sanità (OMS), una percentuale significativa di vaccini viene sprecata ogni anno a causa di interruzioni nella catena del freddo.
I problemi principali sono:
\begin{itemize}
    \item \textbf{Mancanza di visibilità end-to-end}: I dati di transito sono spesso disponibili solo a posteriori.
    \item \textbf{Conflitto di interessi}: Il trasportatore, responsabile del mantenimento della temperatura, è spesso anche colui che fornisce i dati di monitoraggio, creando un incentivo alla manipolazione in caso di guasti.
    \item \textbf{Silos informativi}: Produttori, distributori e farmacie utilizzano sistemi ERP diversi che non comunicano in tempo reale.
\end{itemize}

\section{Obiettivi del Progetto}
Il sistema proposto mira a risolvere queste problematiche attraverso tre pilastri fondamentali:

\subsection{1. Automazione tramite Smart Contract}
Eliminare l'intermediazione umana e burocratica nei processi di verifica e pagamento. Il contratto intelligente (Smart Contract) agisce come un deposito a garanzia (Escrow), sbloccando i fondi al corriere solo se tutte le condizioni di qualità sono matematicamente soddisfatte.

\subsection{2. Sicurezza del Dato (Data Integrity)}
Garantire che, una volta acquisito, il dato non possa essere alterato (Tamper-Proof). Questo è assicurato dalla crittografia sottostante la blockchain e dal meccanismo di consenso IBFT 2.0 di Hyperledger Besu.

\subsection{3. Validazione Logica (Data Validity)}
Garantire che il dato acquisito rifletta la realtà. Qui interviene l'Oracolo Bayesiano, che correla letture diverse (es. "Temperatura Alta" + "Sigillo Rotto" + "Luce Rilevata") per calcolare la probabilità posteriore di un evento avverso, distinguendo tra falsi positivi dei sensori e reali compromissioni del carico.