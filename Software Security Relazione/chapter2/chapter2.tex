\chapter{Analisi e Progettazione Architetturale}

\begin{preamble}
Questo capitolo dettaglia le scelte architetturali adottate per soddisfare i requisiti di sicurezza. Si analizzano le tecnologie selezionate (Hyperledger Besu, Truffle), il design del sistema distribuito e l'architettura a tre livelli (Blockchain, Oracle Middleware, Web UI).
\end{preamble}

\section{Architettura Distribuita a Tre Livelli}
Il sistema è basato su un'architettura decentralizzata che interagisce con componenti off-chain per garantire usabilità e connessione con il mondo fisico.

\subsection{Livello Blockchain (Data \& Logic Layer)}
Il cuore del sistema è una rete privata basata su \textbf{Hyperledger Besu}.
\begin{itemize}
    \item \textbf{Ruolo}: Mantiene il registro immutabile delle transazioni (Ledger) e ospita la logica di business (Smart Contracts).
    \item \textbf{Componenti}: Nodi validatori, Smart Contract \texttt{BNCore}, \texttt{BNGestoreSpedizioni} e \texttt{BNPagamenti}.
\end{itemize}

\subsection{Livello Middleware (Oracle Layer)}
Poiché la blockchain è un sistema chiuso che non può accedere a dati esterni (internet/sensori), è necessario un componente "ponte".
\begin{itemize}
    \item \textbf{Componente}: Script Node.js \texttt{simula\_oracolo.js}.
    \item \textbf{Funzione}: Questo script agisce da bridge. Simula l'acquisizione dati dai sensori IoT (Temperatura, Umidità, Shock, Luce, Sigillo), esegue una pre-validazione opzionale e invia le "evidenze" allo Smart Contract tramite transazioni firmate dal \texttt{RUOLO\_SENSORE}.
\end{itemize}

\subsection{Livello Presentazione (Web Interface)}
L'interfaccia utente permette agli attori umani di interagire col sistema senza dover usare riga di comando.
\begin{itemize}
    \item \textbf{Tecnologia}: Single Page Application (SPA) HTML5/JS connessa via Web3.js.
    \item \textbf{Ruolo}: Dashboard per la creazione spedizioni, monitoraggio real-time e gestione rimborsi.
\end{itemize}

\section{Focus Tecnologico: Hyperledger Besu}
La scelta di Hyperledger Besu rispetto ad altre soluzioni (es. Geth, Hyperledger Fabric) è stata guidata da specifici requisiti di sicurezza enterprise.

\subsection{Consenso IBFT 2.0 (Istanbul Byzantine Fault Tolerance)}
A differenza del Proof-of-Work (costoso e lento) o Proof-of-Authority semplice, IBFT 2.0 offre:
\begin{itemize}
    \item \textbf{Finalità Immediata}: Una volta che un blocco è scritto, non può essere riorganizzato (niente "fork"). Questo è critico per la supply chain: una consegna registrata non può "sparire".
    \item \textbf{Tolleranza ai Guasti Bizantini}: Il sistema continua a funzionare correttamente anche se fino a $f$ nodi su $N$ sono malevoli o offline, dove $N \ge 3f + 1$. Nella nostra configurazione a 4 nodi, il sistema resiste alla compromissione completa di 1 nodo validatore senza perdere integrità o disponibilità.
\end{itemize}

\subsection{Permissioning Avanzato}
Besu permette di definire una "Allowlist" di nodi e account a livello di protocollo.
\begin{itemize}
    \item \textbf{Node Whitelisting}: Solo i nodi certificati (es. appartenenti a Produttore e Distributore) possono partecipare al consenso e sincronizzare la blockchain.
    \item \textbf{Smart Contract Permissions}: L'accesso alle funzioni critiche è limitato a livello applicativo (tramite libreria AccessControl di OpenZeppelin), distinguendo ruoli come \texttt{ADMIN}, \texttt{MITTENTE} e \texttt{SENSORE}.
\end{itemize}

\section{Principi di Design Sicuro (Saltzer \& Schroeder)}
L'architettura rispetta i principi fondamentali della sicurezza:
\begin{itemize}
    \item \textbf{Economy of Mechanism (Semplicità)}: I contratti sono modulari. \texttt{BNCore} fa solo matematica, \texttt{BNGestore} gestisce i processi. Meno codice = meno bug.
    \item \textbf{Open Design}: La sicurezza non si basa sull'oscurità. Il codice è pubblico e verificabile; la sicurezza deriva dalla crittografia e dalla matematica del consenso.
    \item \textbf{Fail-Safe Defaults}: Se una condizione non è verificata (es. evidenze mancanti), lo stato di default è "Blocco dei fondi" o "Rifiuto transazione", mai "Accettazione implicita".
    \item \textbf{Separation of Privilege}: Per sbloccare un pagamento servono due condizioni distinte: l'invio delle evidenze (dal Sensore) e la verifica probabilistica (dal Contratto). Nessun singolo attore ha il potere totale.
\end{itemize}

\section{Analisi di Resistenza, Sopravvivenza e Ambiguità}
\begin{itemize}
    \item \textbf{Resistenza}: L'uso di crittografia asimmetrica rende impossibile la falsificazione delle firme digitali dei sensori.
    \item \textbf{Sopravvivenza (Resilienza)}: La natura distribuita del ledger assicura che i dati siano replicati su tutti i nodi. Un attacco DDoS verso un singolo nodo non ferma il servizio.
    \item \textbf{Ambiguità (Obfuscation/Privacy)}: Il sistema adotta un approccio ibrido "Privacy by Design". 
    \begin{itemize}
        \item \textit{Logica Pubblica}: I calcoli probabilistici sono trasparenti per garantire l'audit.
        \item \textit{Dati Sensibili Offuscati}: I dettagli personali (nomi, lotti farmaceutici) non sono salvati in chiaro on-chain. Viene memorizzato solo un \textbf{Hash crittografico} (`hashedDetails`) che permette la verifica di integrità senza rivelare il contenuto a osservatori non autorizzati (Off-chain Data Storage).
    \end{itemize}
\end{itemize}
