\chapter{Verifica, Validazione e Modellazione Formale}

\begin{preamble}
In questo capitolo vengono esposti i risultati delle attività di verifica e validazione. Si descrivono gli esiti dell'analisi statica del codice, i test funzionali eseguiti sulla rete Hyperledger Besu e, in particolare, la modellazione formale di unit critiche tramite Catene di Markov (PRISM) per la verifica di proprietà di Safety e Guarantee.
\end{preamble}

\section{Analisi Statica e Audit}
Il codice è stato sottoposto ad analisi statica automatizzata per identificare vulnerabilità note e difetti di conformità.

\subsection{Risultati Solidity Analyzer e Solhint}
\begin{itemize}
    \item \textbf{Solidity Analyzer (Remix)}: L'analisi ha confermato l'assenza di vulnerabilità critiche come Reentrancy, Integer Overflow (mitigato da Solidity 0.8+) e Unchecked Call Return Values. Sono stati risolti warning relativi a visibilità delle funzioni e gas costs.
    \item \textbf{Solhint}: Il codice rispetta le regole di stile configurate, garantendo coerenza nell'indentazione e naming convention. Tutte le segnalazioni di priorità "Error" sono state corrette.
\end{itemize}

\section{Verifica Formale con PRISM}
Per garantire la robustezza logica del sistema rispetto alle minacce identificate (Spoofing e Tampering), è stato modellato il comportamento probabilistico dell'unità "Sensore-Oracolo" utilizzando il model checker \textbf{PRISM}. L'analisi si basa su Catene di Markov a Tempo Discreto (DTMC) e segue una metodologia comparativa: viene prima analizzato il sistema vulnerabile (senza contromisure) e successivamente il sistema protetto (con Active Defense, TPM e Ridondanza).

\subsection{Introduzione: Obiettivo della Modellazione}

\subsubsection{Contesto del Sistema}
Il sistema oggetto di questa analisi è un sistema di monitoraggio IoT per la supply chain composto da cinque sensori critici che monitorano lo stato di merci durante il trasporto. L'architettura del sistema comprende il sensore E1 per il monitoraggio della temperatura, il sensore E2 per il controllo del sigillo, il sensore E3 per il rilevamento degli shock meccanici, il sensore E4 per il monitoraggio della luce, e il sensore E5 per la scansione all'arrivo. Questi componenti sono stati progettati per garantire l'integrità e la tracciabilità dei dati durante l'intera catena logistica.

\subsubsection{Scopo dell'Analisi di Markov Chain}
L'obiettivo di questa analisi è modellare formalmente il comportamento probabilistico del sistema di sensori utilizzando Discrete-Time Markov Chains (DTMC). L'approccio metodologico persegue tre obiettivi principali: quantificare l'efficacia delle contromisure di sicurezza implementate attraverso l'analisi DUAL-STRIDE, verificare formalmente proprietà di Safety e Guarantee/Response utilizzando il model checker PRISM, e confrontare quantitativamente il sistema con e senza contromisure per dimostrare l'impatto delle misure di sicurezza adottate.

\subsubsection{Minacce Modellate}
L'analisi DUAL-STRIDE ha identificato come critiche per l'integrità del sistema due principali minacce appartenenti alla tassonomia STRIDE. La prima è Spoofing (S2.1), in cui un sensore falso può iniettare dati malevoli nel sistema compromettendone l'autenticità. La seconda è Tampering (T2.1), che consiste nella manomissione fisica dei sensori con conseguente alterazione delle letture. Tali minacce rappresentano vettori di attacco significativi che possono compromettere la confidenzialità e l'integrità dei dati raccolti.

\subsubsection{Contromisure Implementate}
Il sistema di sicurezza implementato si basa su tre pilastri fondamentali. Il primo consiste in Device Attestation basato su Trusted Platform Module (TPM) combinato con Mutual TLS, che fornisce autenticazione bilaterale e blocca efficacemente attacchi di tipo Spoofing. Il secondo pilastro è rappresentato dalla Sensor Redundancy, che mitiga i rischi derivanti da Tampering fisico mediante ridondanza hardware. Il terzo pilastro è costituito dall'Active Defense System, un sistema di difesa attivo composto da tre componenti: Intrusion Detection System (IDS) per il rilevamento dei tentativi di attacco, Rate Limiting per il conteggio dei fallimenti di autenticazione, e System Lock che blocca permanentemente il sensore dopo tre tentativi di attacco consecutivi.

\subsection{Modello PRISM: Sistema SENZA Contromisure}

\subsubsection{Struttura del Modello}
Il modello PRISM rappresenta il sistema prima dell'implementazione delle contromisure DUAL-STRIDE, evidenziando la vulnerabilità intrinseca agli attacchi. Il modello è dichiarato come Discrete-Time Markov Chain (DTMC), in cui il tempo avanza in step discreti e le transizioni tra stati seguono distribuzioni probabilistiche definite.

\paragraph{Dichiarazione del Tipo di Modello}
\begin{lstlisting}[style=mystyle, language=Prolog]
dtmc
\end{lstlisting}
Questa dichiarazione specifica che il modello adotta una Discrete-Time Markov Chain (DTMC), dove il tempo avanza in step discreti e le transizioni sono governate da probabilità.

\paragraph{Variabili di Stato (Senza Active Defense)}
\begin{lstlisting}[style=mystyle, language=Prolog]
module sensor_system_vulnerable
    
    e1 : [0..2] init 0;  // Sensore E1: Temperatura
    e2 : [0..2] init 0;  // Sensore E2: Sigillo
    e3 : [0..2] init 0;  // Sensore E3: Shock
    e4 : [0..2] init 0;  // Sensore E4: Luce
    e5 : [0..2] init 0;  // Sensore E5: Scan Arrivo
    
    time : [0..200] init 0;
\end{lstlisting}

Ogni sensore è modellato attraverso una variabile di stato con dominio [0..2], dove lo stato 0 corrisponde al sensore funzionante e sicuro (OK), lo stato 1 rappresenta un guasto hardware non derivante da compromissione (FAILED), e lo stato 2 indica un sensore sotto attacco riuscito (COMPROMISED). Il modello include inoltre un contatore temporale che avanza da 0 a 200 step, definendo l'orizzonte temporale dell'analisi.
Una caratteristica distintiva di questo modello vulnerabile è l'assenza del contatore \texttt{e1\_attempts} e del flag \texttt{e1\_locked}, indicando che non sono implementati meccanismi di IDS, Rate Limiting o Active Defense. Tutti i sensori inizializzano nello stato OK (\texttt{init 0}) per consentire un confronto equo con il modello protetto.

\subsubsection{Matrice di Transizione: Sistema SENZA Contromisure}
La seguente matrice mostra le probabilità di transizione per un singolo sensore senza contromisure:

\begin{table}[H]
\centering
\begin{tabularx}{\textwidth}{|l|X|X|X|}
\hline
\textbf{Da Stato $\downarrow$ / A Stato $\rightarrow$} & \textbf{OK (0)} & \textbf{FAILED (1)} & \textbf{COMPR. (2)} \\ \hline
\textbf{OK (0)} & 0.80 & 0.05 & \textbf{0.15} \\ \hline
\textbf{FAILED (1)} & 0.60 & 0.30 & \textbf{0.10} \\ \hline
\textbf{COMPR. (2)} & 0.00 & 0.00 & 1.00 \\ \hline
\end{tabularx}
\caption{Matrice di Transizione - Sistema Vulnerabile}
\end{table}

Dallo stato OK, il sensore ha una probabilità dell'80\% di rimanere operativo in assenza di eventi, una probabilità del 5\% di transizione verso lo stato FAILED dovuta a guasti hardware naturali, e una probabilità critica del 15\% di transizione verso lo stato COMPROMISED, dovuta ad attacchi riusciti (Spoofing 5\% + Tampering 10\%). Dallo stato FAILED, il recovery manuale presenta una probabilità del 60\%, mentre vi è una probabilità del 30\% che il sensore rimanga guasto e una preoccupante probabilità del 10\% di compromissione, indicando una maggiore vulnerabilità dei sensori in stato di guasto. Lo stato COMPROMISED costituisce uno stato assorbente.

Le osservazioni critiche evidenziano un'alta probabilità di compromissione (15\% da OK e 10\% da FAILED), la natura assorbente dello stato compromesso che impedisce qualsiasi forma di recupero, e un recovery lento caratterizzato da una probabilità di solo 60\% dalla condizione di guasto.

\paragraph{Diagramma di Stati: Sistema SENZA Contromisure (Rappresentazione Testuale)}
\begin{enumerate}
    \item \textbf{OK}: Stato operativo normale. Transita a:
    \begin{itemize}
        \item OK (80\%): Nessun evento.
        \item FAILED (5\%): Guasto naturale.
        \item COMPROMISED (15\%): Attacco riuscito.
    \end{itemize}
    \item \textbf{FAILED}: Guasto hardware. Transita a:
    \begin{itemize}
        \item OK (60\%): Recovery manuale.
        \item FAILED (30\%): Rimane guasto.
        \item COMPROMISED (10\%): Attacco su sensore guasto.
    \end{itemize}
    \item \textbf{COMPROMISED}: Stato assorbente. Transita a:
    \begin{itemize}
        \item COMPROMISED (100\%): Nessun recovery possibile.
    \end{itemize}
\end{enumerate}

\subsubsection{Logica delle Transizioni: Sistema Vulnerabile}

\paragraph{Sensore OK $\rightarrow$ OK, FAILED, o COMPROMISED}
\begin{lstlisting}[style=mystyle, language=Prolog]
    [] e1=0 & time<200 -> 
        0.80 : (e1'=0) & (time'=time+1) +     // Rimane OK
        0.05 : (e1'=1) & (time'=time+1) +     // Guasto naturale
        0.15 : (e1'=2) & (time'=time+1);      // ATTACCO RIUSCITO
\end{lstlisting}
Questa regola di transizione codifica tre possibili esiti per un sensore nello stato OK. Con probabilità 80\% il sensore rimane operativo, con probabilità 5\% si verifica un guasto hardware naturale, e con probabilità 15\% si verifica un attacco riuscito che porta il sensore allo stato COMPROMISED. Quest'ultima transizione rappresenta il punto critico del sistema vulnerabile.

\paragraph{Sensore FAILED $\rightarrow$ OK, FAILED, o COMPROMISED}
\begin{lstlisting}[style=mystyle, language=Prolog]
    [] e1=1 & time<200 -> 
        0.60 : (e1'=0) & (time'=time+1) +     // Recovery manuale
        0.30 : (e1'=1) & (time'=time+1) +     // Rimane guasto
        0.10 : (e1'=2) & (time'=time+1);      // ATTACCO (piu' vulnerabile)
\end{lstlisting}
Per un sensore in stato FAILED, la regola modella il processo di recovery manuale che, in assenza di meccanismi di Auto-Failover, presenta un tasso di successo del 60\%, inferiore rispetto al modello protetto.

\paragraph{Sensore COMPROMISED $\rightarrow$ COMPROMISED (Stato Assorbente)}
\begin{lstlisting}[style=mystyle, language=Prolog]
    [] e1=2 & time<200 -> 
        1.00 : (e1'=2) & (time'=time+1);      // Rimane compromesso
\end{lstlisting}
Lo stato COMPROMISED è modellato come stato assorbente mediante una probabilità unitaria di auto-transizione.

\subsubsection{Formule Derivate}
\begin{lstlisting}[style=mystyle, language=Prolog]
formula num_ok = (e1=0?1:0) + (e2=0?1:0) + (e3=0?1:0) + (e4=0?1:0) + (e5=0?1:0);
formula num_failed = (e1=1?1:0) + (e2=1?1:0) + (e3=1?1:0) + (e4=1?1:0) + (e5=1?1:0);
formula num_compromised = (e1=2?1:0) + (e2=2?1:0) + (e3=2?1:0) + (e4=2?1:0) + (e5=2?1:0);

formula is_system_compromised = (num_compromised >= 1);
formula is_system_operational = (num_ok = 5);
formula is_system_degraded = (num_failed >= 1) & !is_system_compromised;
formula is_safe = !is_system_compromised;
\end{lstlisting}
Il modello definisce formule ausiliarie per classificare lo stato aggregato del sistema. In particolare, il sistema è considerato compromesso se almeno un sensore si trova nello stato COMPROMISED, riflettendo l'assenza di Sensor Redundancy nel modello vulnerabile.

\subsection{Proprietà PCTL Verificate: Sistema SENZA Contromisure}

\subsubsection{Proprietà di Safety (S1)}
\paragraph{Codice PCTL}
\begin{lstlisting}[style=mystyle, language=Prolog]
P=? [ G<=100 (e1!=2 & e2!=2 & e3!=2 & e4!=2 & e5!=2) ]
\end{lstlisting}
\paragraph{Risultato PRISM}
\textbf{Risultato}: $1.4877 \times 10^{-7}$ $\approx$ \textbf{0.0000149\%}
\paragraph{Analisi}
Il risultato della verifica PRISM evidenzia la vulnerabilità critica del sistema non protetto. La probabilità che il sistema non venga compromesso in 100 step è praticamente nulla.

\subsubsection{Proprietà di Guarantee/Response (G1)}
\paragraph{Codice PCTL}
\begin{lstlisting}[style=mystyle, language=Prolog]
P=? [ F<=20 (e1=0 & e2=0 & e3=0 & e4=0 & e5=0) ]
\end{lstlisting}
\paragraph{Risultato PRISM}
\textbf{Risultato}: $0.435146$ $\approx$ \textbf{43.5\%}
\paragraph{Analisi}
Il risultato evidenzia significative limitazioni nel processo di recovery del sistema vulnerabile. In assenza di meccanismi di Auto-Failover, il recovery è manuale e lento.

\subsection{Modello PRISM: Sistema CON Contromisure}

\subsubsection{Struttura del Modello}
Il modello PRISM rappresenta il sistema con tutte le contromisure di sicurezza attive. Include variabili di stato aggiuntive per implementare l'Active Defense System.

\paragraph{Variabili di Stato del Sensore E1 (con Active Defense)}
\begin{lstlisting}[style=mystyle, language=Prolog]
module sensor_system_active_defense
    
    e1 : [0..2] init 1;           // 0=OK, 1=FAILED, 2=COMPROMISED 
    e1_attempts : [0..3] init 0;  // Contatore tentativi di attacco
    e1_locked : bool init false;  // Stato di blocco difensivo
\end{lstlisting}
Per il sensore E1, il modello definisce tre variabili che estendono la rappresentazione base. \texttt{e1\_attempts} funge da contatore per i tentativi di attacco rilevati dall'IDS. \texttt{e1\_locked} indica se il sistema ha attivato il blocco di sicurezza.

\subsubsection{Matrice di Transizione: Sistema CON Contromisure}
La seguente matrice mostra le probabilità di transizione tra gli stati per un sensore con contromisure attive:

\begin{table}[H]
\centering
\begin{tabularx}{\textwidth}{|l|X|X|X|}
\hline
\textbf{Da Stato $\downarrow$ / A Stato $\rightarrow$} & \textbf{OK (0)} & \textbf{FAILED (1)} & \textbf{COMPR. (2)} \\ \hline
\textbf{OK (0)} & 0.90 & 0.05 & \textbf{0.00} \\ \hline
\textbf{FAILED (1)} & 0.95 & 0.05 & \textbf{0.00} \\ \hline
\textbf{COMPR. (2)} & 0.00 & 0.00 & 1.00 \\ \hline
\end{tabularx}
\caption{Matrice di Transizione - Sistema Protetto}
\end{table}

Dallo stato OK, il sensore ha una probabilità del 90\% di rimanere operativo, una probabilità del 5\% di guasto hardware naturale, e una probabilità nulla di compromissione. Dallo stato FAILED, il recovery automatico tramite Auto-Failover presenta un'alta probabilità di successo del 95\%.

\subsubsection{Logica delle Transizioni: Active Defense}

\paragraph{CASO 1: Sensore Normale (OK, Non Bloccato)}
\begin{lstlisting}[style=mystyle, language=Prolog]
    [] e1=0 & !e1_locked & e1_attempts < 3 & time<200 -> 
        0.90 : (e1'=0) & (time'=time+1) +                                     // Nessun evento
        0.05 : (e1'=1) & (time'=time+1) +                                     // Guasto naturale
        0.05 : (e1'=0) & (e1_attempts'=e1_attempts+1) & (time'=time+1);      // ATTACCO RILEVATO
\end{lstlisting}
Questa regola è cruciale: l'attaccante tenta un attacco (5\%), ma l'IDS lo rileva, le contromisure lo bloccano, il sensore rimane OK e il contatore dei tentativi viene incrementato.

\paragraph{CASO 2: System Lock (Dopo 3 Tentativi)}
\begin{lstlisting}[style=mystyle, language=Prolog]
    [] e1=0 & !e1_locked & e1_attempts = 3 & time<200 ->
        1.00 : (e1_locked'=true) & (time'=time+1);                            // ATTIVA BLOCCO
\end{lstlisting}
Quando il contatore raggiunge tre tentativi di attacco bloccati, si attiva il System Lock.

\subsection{Proprietà PCTL Verificate: Sistema CON Contromisure}

\subsubsection{Proprietà di Safety (S1)}
\paragraph{Codice PCTL}
\begin{lstlisting}[style=mystyle, language=Prolog]
P=? [ G<=100 (e1!=2 & e2!=2 & e3!=2 & e4!=2 & e5!=2) ]
\end{lstlisting}
\paragraph{Risultato PRISM}
\textbf{Risultato}: \textbf{1.0} (100\%)
\paragraph{Analisi}
Il risultato costituisce una dimostrazione formale dell'efficacia delle contromisure. La probabilità del 100\% di non-compromissione evidenzia che le contromisure bloccano completamente gli attacchi.

\subsubsection{Proprietà di Guarantee/Response (G1)}
\paragraph{Codice PCTL}
\begin{lstlisting}[style=mystyle, language=Prolog]
P=? [ F<=20 (e1=0 & e2=0 & e3=0 & e4=0 & e5=0) ]
\end{lstlisting}
\paragraph{Risultato PRISM}
\textbf{Risultato}: $\approx$ \textbf{97\%}
\paragraph{Analisi}
Il risultato dimostra l'alta efficacia dei meccanismi di Sensor Redundancy e Auto-Failover.

\subsubsection{Proprietà di Active Defense Verification}
\paragraph{Codice PCTL}
\begin{lstlisting}[style=mystyle, language=Prolog]
P=? [ F e1_locked ]
\end{lstlisting}
\paragraph{Analisi}
Conferma che l'IDS e il Rate Limiting funzionano correttamente.

\subsection{Confronto Quantitativo: Con vs Senza Contromisure}

\subsubsection{Confronto delle Matrici di Transizione}
\begin{table}[H]
\centering
\begin{tabularx}{\textwidth}{|l|X|X|X|}
\hline
\textbf{Transizione} & \textbf{CON Contromisure} & \textbf{SENZA Contromisure} & \textbf{Differenza} \\ \hline
OK $\rightarrow$ OK & 90\% & 80\% & +10\% \\ \hline
OK $\rightarrow$ FAILED & 5\% & 5\% & 0\% \\ \hline
OK $\rightarrow$ COMPR. & \textbf{0\%} & \textbf{15\%} & \textbf{-15\%} \\ \hline
FAILED $\rightarrow$ OK & \textbf{95\%} & \textbf{60\%} & \textbf{+35\%} \\ \hline
FAILED $\rightarrow$ FAILED & 5\% & 30\% & -25\% \\ \hline
FAILED $\rightarrow$ COMPR. & \textbf{0\%} & \textbf{10\%} & \textbf{-10\%} \\ \hline
\end{tabularx}
\caption{Confronto Matrici di Transizione}
\end{table}

\subsubsection{Tabella Comparativa dei Risultati PRISM}
\begin{table}[H]
\centering
\begin{tabularx}{\textwidth}{|X|X|X|X|}
\hline
\textbf{Proprietà} & \textbf{CON C.} & \textbf{SENZA C.} & \textbf{Miglior.} \\ \hline
Safety (S1) & \textbf{100\%} & 0.00001\% & +99.9\% \\ \hline
Guarantee (G1) & \textbf{97\%} & 43.5\% & +53.5\% \\ \hline
\end{tabularx}
\caption{Confronto Risultati PRISM}
\end{table}

\subsection{Conclusioni Analisi Formale}
L'analisi di Markov Chain condotta mediante il model checker PRISM ha fornito una dimostrazione formale dell'efficacia delle contromisure di sicurezza. Per la proprietà di Safety, le contromisure riducono la vulnerabilità da approssimativamente il 100\% allo 0\%. Per la proprietà di Guarantee/Response, le contromisure migliorano la probabilità di recovery dal 43.5\% al 97\%. Il meccanismo di Active Defense si è dimostrato efficace nel rilevare e bloccare attacchi persistenti.


\section{Testing su Blockchain Privata (Besu)}
Tutti i componenti sono stati integrati e testati in un ambiente reale basato su Hyperledger Besu.

\subsection{Ambienti di Test}
\begin{itemize}
    \item \textbf{Unit Testing}: Suite completa di test JavaScript (Framework Truffle/Mocha) eseguita su Ganache per test rapidi della logica.
    \item \textbf{Integration Testing}: Deployment su rete privata Besu a 4 nodi (consenso IBFT 2.0). Sono stati verificati simulando scenari di latenza di rete e spegnimento di un nodo validatore.
    \item \textbf{Privacy Compliance}: Eseguiti test specifici (\texttt{test-offuscamento.js}) per validare che le tabelle CPT siano accessibili solo dall'Admin e che i dettagli sensibili siano verificabili solo tramite hash, impedendo letture non autorizzate.
\end{itemize}

\subsection{Simulazione con Oracolo Scriptato}
Utilizzando lo script \texttt{simula\_oracolo.js}, è stato possibile testare il comportamento del sistema su un campione di N=1000 iterazioni simulate.
\begin{itemize}
    \item \textbf{Scenario 1 (Condizioni Normali)}: Con sensori che riportano valori nominali (90\% dei casi), la Rete Bayesiana On-Chain ha correttamente valutato la probabilità di conformità > 95\% nel 100\% dei casi.
    \item \textbf{Scenario 2 (Manomissione)}: Forzando il sensore "Sigillo" a \texttt{False}, la probabilità calcolata dal contratto \texttt{BNCore} è scesa immediatamente sotto la soglia di sicurezza, attivando lo stato di allarme.
\end{itemize}

\subsection{Risultati}
I test hanno dimostrato che il sistema mantiene la consistenza dei dati anche con un nodo offline. Le transazioni vengono confermate e finalizzate correttamente grazie al consenso IBFT. I monitor di runtime hanno intercettato correttamente il 100\% delle transazioni anomale simulate (es. tentativi di registrare temperature fuori range senza triggerare allarmi).
