\chapter{Verifica, Validazione e Modellazione Formale}

\begin{preamble}
In questo capitolo vengono esposti i risultati delle attività di verifica e validazione. Si descrivono gli esiti dell'analisi statica del codice, i test funzionali eseguiti sulla rete Hyperledger Besu e, in particolare, la modellazione formale di unit critiche tramite Catene di Markov (PRISM) per la verifica di proprietà di Safety e Guarantee.
\end{preamble}

\section{Analisi Statica e Audit}
Il codice è stato sottoposto ad analisi statica automatizzata per identificare vulnerabilità note e difetti di conformità.

\subsection{Risultati Solidity Analyzer e Solhint}
\begin{itemize}
    \item \textbf{Solidity Analyzer (Remix)}: L'analisi ha confermato l'assenza di vulnerabilità critiche come Reentrancy, Integer Overflow (mitigato da Solidity 0.8+) e Unchecked Call Return Values. Sono stati risolti warning relativi a visibilità delle funzioni e gas costs.
    \item \textbf{Solhint}: Il codice rispetta le regole di stile configurate, garantendo coerenza nell'indentazione e naming convention. Tutte le segnalazioni di priorità "Error" sono state corrette.
\end{itemize}

\section{Verifica Formale con PRISM}
Per garantire la robustezza logica del sistema rispetto alle minacce identificate (Spoofing e Tampering), è stato modellato il comportamento probabilistico dell'unità "Sensore-Oracolo" utilizzando il model checker \textbf{PRISM}. L'analisi si basa su Catene di Markov a Tempo Discreto (DTMC) e segue una metodologia comparativa: viene prima analizzato il sistema vulnerabile (senza contromisure) e successivamente il sistema protetto (con Active Defense, TPM e Ridondanza).

\subsection{Obiettivi e Minacce Modellate}
L'obiettivo è quantificare l'efficacia delle contromisure nel mitigare due minacce critiche STRIDE:
\begin{itemize}
    \item \textbf{Spoofing (S2.1)}: Un sensore falso inietta dati malevoli (Probabilità stimata: 5\% per step).
    \item \textbf{Tampering (T2.1)}: Manomissione fisica del sensore (Probabilità stimata: 10\% per step).
\end{itemize}
Il modello verifica proprietà di \textit{Safety} (Probabilità di compromissione) e \textit{Guarantee/Response} (Capacità di recovery).

\subsection{Analisi 1: Sistema Vulnerabile (Senza Contromisure)}
Il modello PRISM del sistema non protetto evidenzia la vulnerabilità intrinseca agli attacchi.

\subsubsection{Matrice di Transizione (Vulnerabile)}
In assenza di difese, le probabilità di transizione per un singolo sensore sono:

\begin{table}[h]
\centering
\begin{tabularx}{\textwidth}{|l|X|X|X|}
\hline
\textbf{Da / A} & \textbf{OK (0)} & \textbf{FAILED (1)} & \textbf{COMPROMISED (2)} \\ \hline
\textbf{OK} & 80\% (Normale) & 5\% (Guasto) & \textbf{15\% (Attacco Riuscito)} \\ \hline
\textbf{FAILED} & 60\% (Recovery Lento) & 30\% (Guasto) & \textbf{10\% (Attacco su Guasto)} \\ \hline
\textbf{COMPR.} & 0\% & 0\% & 100\% (Stato Assorbente) \\ \hline
\end{tabularx}
\caption{Matrice di Transizione - Sistema Vulnerabile}
\end{table}

\textbf{Criticità}:
\begin{enumerate}
    \item \textbf{Alta probabilità di compromissione}: 15\% ad ogni step dallo stato OK.
    \item \textbf{Recovery Lento}: Solo il 60\% di probabilità di ripristino da un guasto (processo manuale).
    \item \textbf{Irreversibilità}: Lo stato \texttt{COMPROMISED} è assorbente. Una volta violato, il sistema è perso.
\end{enumerate}

\subsubsection{Risultati Verifica Formale (Vulnerabile)}
\begin{enumerate}
    \item \textbf{Safety (S1)}: \texttt{P=? [ G<=100 (nessun\_sensore\_compromesso) ]} \\
    \textbf{Risultato}: $1.49 \times 10^{-7}$ ($\approx 0\%$).
    \\ \textit{Analisi}: La compromissione è matematicamente certa entro 100 step.
    
    \item \textbf{Guarantee (G1)}: \texttt{P=? [ F<=20 (tutti\_sensori\_OK) ]} \\
    \textbf{Risultato}: $43.5\%$.
    \\ \textit{Analisi}: Senza auto-failover, il sistema fatica a recuperare la piena operatività in tempi utili.
\end{enumerate}

\subsection{Analisi 2: Sistema Protetto (Con Contromisure)}
Il modello protetto implementa:
\begin{itemize}
    \item \textbf{Device Attestation (TPM) + mTLS}: Blocca lo Spoofing.
    \item \textbf{Sensor Redundancy}: Mitiga il Tampering.
    \item \textbf{Active Defense}: IDS che conta i tentativi. Al 3° tentativo bloccato, il sistema va in \texttt{LOCKED}.
\end{itemize}

\subsubsection{Matrice di Transizione (Protetto)}
Le contromisure modificano radicalmente le probabilità:

\begin{table}[h]
\centering
\begin{tabularx}{\textwidth}{|l|X|X|X|}
\hline
\textbf{Da / A} & \textbf{OK (0)} & \textbf{FAILED (1)} & \textbf{COMPROMISED (2)} \\ \hline
\textbf{OK} & 90\% (Protetto) & 5\% (Guasto) & \textbf{0\% (Bloccato)} \\ \hline
\textbf{FAILED} & \textbf{95\% (Auto-Failover)} & 5\% (Guasto) & \textbf{0\% (Bloccato)} \\ \hline
\textbf{COMPR.} & 0\% & 0\% & 100\% (Irraggiungibile) \\ \hline
\end{tabularx}
\caption{Matrice di Transizione - Sistema Protetto}
\end{table}

\textbf{Meccanismo Active Defense}:
\begin{lstlisting}[language=Prolog, basicstyle=\tiny\ttfamily]
[] e1=0 & attempts < 3 -> 
    0.90 : (stay_ok) + 0.05 : (fault) + 
    0.05 : (attack_blocked) & (attempts' = attempts + 1); // IDS rileva e blocca

[] attempts = 3 -> 1.00 : (locked' = true); // SYSTEM LOCK
\end{lstlisting}
Gli attacchi avvengono ancora (5\%), ma vengono \textbf{bloccati} e contati. Dopo 3 tentativi, il sensore si "blinda" (Locked).

\subsubsection{Risultati Verifica Formale (Protetto)}
\begin{enumerate}
    \item \textbf{Safety (S1)}: \texttt{P=? [ G<=100 (nessun\_sensore\_compromesso) ]} \\
    \textbf{Risultato}: \textbf{1.0} ($100\%$).
    \\ \textit{Analisi}: Le contromisure eliminano completamente il rischio di compromissione. Lo stato \texttt{COMPROMISED} diviene formalmente irraggiungibile.
    
    \item \textbf{Guarantee (G1)}: \texttt{P=? [ F<=20 (tutti\_sensori\_OK) ]} \\
    \textbf{Risultato}: \textbf{97\%}.
    \\ \textit{Analisi}: L'Auto-Failover garantisce un ripristino rapido (95\% per step) anche in caso di guasti multipli.

    \item \textbf{Active Defense}: \texttt{P=? [ F e1\_locked ]} \\
    \textit{Analisi}: Conferma che il sistema attiva correttamente il blocco di sicurezza in risposta ad attacchi persistenti.
\end{enumerate}

\subsection{Confronto Quantitativo Finale}
La tabella seguente riassume l'impatto delle scelte architetturali sulla sicurezza del sistema.

\begin{table}[h]
\centering
\begin{tabularx}{\textwidth}{@{}l X X r@{}}
\toprule
\textbf{Metrica} & \textbf{Senza Contromisure} & \textbf{Con Contromisure} & \textbf{Delta} \\ \midrule
Safety (100 step) & $\approx 0\%$ & $\mathbf{100\%}$ & $+100\%$ \\
Guarantee (20 step) & $43.5\%$ & $\mathbf{97\%}$ & $+53.5\%$ \\
Transizione a Comp. & $15\%$ (per step) & $\mathbf{0\%}$ & $-15\%$ \\
Recovery Rate & $60\%$ (manuale) & $\mathbf{95\%}$ (auto) & $+35\%$ \\ \bottomrule
\end{tabularx}
\caption{Benchmark di Sicurezza: Vulnerabile vs Protetto}
\label{tab:benchmark_prism}
\end{table}

In conclusione, l'analisi formale dimostra che l'architettura proposta trasforma un sistema intrinsecamente vulnerabile (certezza di compromissione) in uno matematicamente sicuro e altamente disponibile.

\section{Testing su Blockchain Privata (Besu)}
Tutti i componenti sono stati integrati e testati in un ambiente reale basato su Hyperledger Besu.

\subsection{Ambienti di Test}
\begin{itemize}
    \item \textbf{Unit Testing}: Suite completa di test JavaScript (Framework Truffle/Mocha) eseguita su Ganache per test rapidi della logica.
    \item \textbf{Integration Testing}: Deployment su rete privata Besu a 4 nodi (consenso IBFT 2.0). Sono stati verificati simulando scenari di latenza di rete e spegnimento di un nodo validatore.
    \item \textbf{Privacy Compliance}: Eseguiti test specifici (\texttt{test-offuscamento.js}) per validare che le tabelle CPT siano accessibili solo dall'Admin e che i dettagli sensibili siano verificabili solo tramite hash, impedendo letture non autorizzate.
\end{itemize}

\subsection{Simulazione con Oracolo Scriptato}
Utilizzando lo script \texttt{simula\_oracolo.js}, è stato possibile testare il comportamento del sistema su un campione di N=1000 iterazioni simulate.
\begin{itemize}
    \item \textbf{Scenario 1 (Condizioni Normali)}: Con sensori che riportano valori nominali (90\% dei casi), la Rete Bayesiana On-Chain ha correttamente valutato la probabilità di conformità > 95\% nel 100\% dei casi.
    \item \textbf{Scenario 2 (Manomissione)}: Forzando il sensore "Sigillo" a \texttt{False}, la probabilità calcolata dal contratto \texttt{BNCore} è scesa immediatamente sotto la soglia di sicurezza, attivando lo stato di allarme.
\end{itemize}

\subsection{Risultati}
I test hanno dimostrato che il sistema mantiene la consistenza dei dati anche con un nodo offline. Le transazioni vengono confermate e finalizzate correttamente grazie al consenso IBFT. I monitor di runtime hanno intercettato correttamente il 100\% delle transazioni anomale simulate (es. tentativi di registrare temperature fuori range senza triggerare allarmi).
