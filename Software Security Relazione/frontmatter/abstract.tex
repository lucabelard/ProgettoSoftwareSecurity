%\selectlanguage{italian}
\begin{abstract}

La catena del freddo farmaceutica rappresenta un processo critico in cui la garanzia dell'integrità dei prodotti è fondamentale per la salute pubblica. I sistemi tradizionali di monitoraggio spesso soffrono di problemi di centralizzazione, opacità e mancanza di automatismi affidabili. 

Questa relazione propone un'architettura innovativa basata su Blockchain e Intelligenza Artificiale per l'automazione sicura e trasparente della validazione delle spedizioni. La soluzione implementata utilizza Smart Contract su rete Ethereum/Hyperledger Besu per orchestrare il processo di business e integra una Rete Bayesiana on-chain come oracolo decisionale. Questo permette di inferire probabilistically la conformità della spedizione a partire da evidenze parziali o rumorose provenienti da sensori IoT, garantendo che i pagamenti vengano sbloccati solo a fronte di condizioni verificate.

Il sistema è stato progettato seguendo un approccio Security-by-Design, adottando la metodologia DUAL-STRIDE-DUA per l'analisi delle minacce e l'identificazione di vulnerabilità sia intenzionali che accidentali. La validità e la resilienza della soluzione sono state verificate attraverso test distribuiti e analisi statica del codice, dimostrando come l'integrazione di logica probabilistica su blockchain possa offrire un livello superiore di sicurezza e fiducia nei processi logistici critici.

\vspace{1em}

\textbf{Parole chiave}: Blockchain, Ethereum, Smart Contract, Rete Bayesiana, Supply Chain, Sicurezza, Threat Modeling, IoT
\end{abstract} 

\selectlanguage{italian}