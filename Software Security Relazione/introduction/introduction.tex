\phantomsection
\addcontentsline{toc}{chapter}{Introduzione}
\chapter*{Introduzione}
\markboth{Introduzione}{}

La gestione della catena del freddo (Pharmaceutical Cold Chain) rappresenta una delle sfide più critiche nel settore logistico sanitario. Il trasporto di farmaci termosensibili, come vaccini e insulina, richiede il mantenimento rigoroso di specifici range di temperatura (tipicamente 2°C - 8°C) lungo l'intera filiera. Deviazioni anche minime possono compromettere l'efficacia del prodotto, con conseguenze potenzialmente letali per i pazienti e ingenti danni economici per le aziende.

Attualmente, la trasparenza di questo processo è limitata dall'uso di sistemi centralizzati e trust-based, dove le informazioni sono spesso frammentate, cartacee o custodite in silos informatici proprietari. Questo scenario rende difficile ricostruire con certezza la "storia termica" di un lotto e lascia spazio a possibili manipolazioni dei dati per coprire errori logistici o negligenze.

In questo contesto, la tecnologia Blockchain offre un cambio di paradigma fondamentale, passando dalla "fiducia negli attori" alla "fiducia nel protocollo". Attraverso un registro distribuito, immutabile e trasparente, è possibile garantire che ogni misurazione registrata sia autentica e non repudiabile. Tuttavia, la blockchain da sola non può verificare la veridicità del dato fisico prima che venga scritto ("Garbage In, Garbage Out").

Questo progetto propone una soluzione ibrida che integra Hyperledger Besu, una blockchain permissioned adatta a contesti enterprise, con un sistema di **Oracoli Bayesiani**. L'approccio innovativo risiede nell'utilizzare l'inferenza probabilistica on-chain per validare la coerenza delle letture multisensoriali (temperatura, umidità, shock, luce, integrità sigillo) prima di finalizzare la transazione di consegna. In questo modo, il sistema non si limita a registrare i dati, ma agisce come un decisore autonomo capace di accettare o rifiutare un lotto in base a politiche di rischio matematicamente definite.

Questa tesi illustra il design, l'implementazione e la verifica di tale architettura sicura, mettendo in luce come l'integrazione tra DLT (Distributed Ledger Technology) e metodi formali possa elevare gli standard di sicurezza e affidabilità nella logistica farmaceutica 4.0.

\section{Il problema della Catena del Freddo}
La spedizione di medicinali sensibili è un processo ad alto rischio. Secondo l'Organizzazione Mondiale della Sanità (OMS), una percentuale significativa di vaccini viene sprecata ogni anno a causa di interruzioni nella catena del freddo.
I problemi principali sono:
\begin{itemize}
    \item \textbf{Mancanza di visibilità end-to-end}: I dati di transito sono spesso disponibili solo a posteriori.
    \item \textbf{Conflitto di interessi}: Il trasportatore, responsabile del mantenimento della temperatura, è spesso anche colui che fornisce i dati di monitoraggio, creando un incentivo alla manipolazione in caso di guasti.
    \item \textbf{Silos informativi}: Produttori, distributori e farmacie utilizzano sistemi ERP diversi che non comunicano in tempo reale.
\end{itemize}

\section{Obiettivi del Progetto}
Il sistema proposto mira a risolvere queste problematiche attraverso tre pilastri fondamentali:

\subsection{1. Automazione tramite Smart Contract}
Eliminare l'intermediazione umana e burocratica nei processi di verifica e pagamento. Il contratto intelligente (Smart Contract) agisce come un deposito a garanzia (Escrow), sbloccando i fondi al corriere solo se tutte le condizioni di qualità sono matematicamente soddisfatte.

\subsection{2. Sicurezza del Dato (Data Integrity)}
Garantire che, una volta acquisito, il dato non possa essere alterato (Tamper-Proof). Questo è assicurato dalla crittografia sottostante la blockchain e dal meccanismo di consenso IBFT 2.0 di Hyperledger Besu.

\subsection{3. Validazione Logica (Data Validity)}
Garantire che il dato acquisito rifletta la realtà. Qui interviene l'Oracolo Bayesiano, che correla letture diverse (es. "Temperatura Alta" + "Sigillo Rotto" + "Luce Rilevata") per calcolare la probabilità posteriore di un evento avverso, distinguendo tra falsi positivi dei sensori e reali compromissioni del carico.

\begin{figure}[ht!]
    \centering
    \includegraphics[width=0.8\textwidth]{chapter1/image/SR completo con sistema nuovo.png}
    \caption{Panoramica del Sistema di Validazione}
\end{figure}
