\chapter{Introduzione}
\markboth{Introduzione}{}

La gestione della catena del freddo (Pharmaceutical Cold Chain) rappresenta una delle sfide più critiche nel settore logistico sanitario. Il trasporto di farmaci termosensibili, come vaccini e insulina, richiede il mantenimento rigoroso di specifici range di temperatura (tipicamente 2°C - 8°C) lungo l'intera filiera. Deviazioni anche minime possono compromettere l'efficacia del prodotto, con conseguenze potenzialmente letali per i pazienti e ingenti danni economici per le aziende.

Attualmente, la trasparenza di questo processo è limitata dall'uso di sistemi centralizzati e trust-based, dove le informazioni sono spesso frammentate, cartacee o custodite in silos informatici proprietari. Questo scenario rende difficile ricostruire con certezza la "storia termica" di un lotto e lascia spazio a possibili manipolazioni dei dati per coprire errori logistici o negligenze.

In questo contesto, la tecnologia Blockchain offre un cambio di paradigma fondamentale, passando dalla "fiducia negli attori" alla "fiducia nel protocollo". Attraverso un registro distribuito, immutabile e trasparente, è possibile garantire che ogni misurazione registrata sia autentica e non repudiabile. Tuttavia, la blockchain da sola non può verificare la veridicità del dato fisico prima che venga scritto ("Garbage In, Garbage Out").

Questo progetto propone una soluzione ibrida che integra Hyperledger Besu, una blockchain permissioned adatta a contesti enterprise, con un sistema di **Oracoli Bayesiani**. L'approccio innovativo risiede nell'utilizzare l'inferenza probabilistica on-chain per validare la coerenza delle letture multisensoriali (temperatura, umidità, shock, luce, integrità sigillo) prima di finalizzare la transazione di consegna. In questo modo, il sistema non si limita a registrare i dati, ma agisce come un decisore autonomo capace di accettare o rifiutare un lotto in base a politiche di rischio matematicamente definite.

Questa tesi illustra il design, l'implementazione e la verifica di tale architettura sicura, mettendo in luce come l'integrazione tra DLT (Distributed Ledger Technology) e metodi formali possa elevare gli standard di sicurezza e affidabilità nella logistica farmaceutica 4.0.

\section{Il problema della Catena del Freddo}
La spedizione di medicinali sensibili è un processo ad alto rischio. Secondo l'Organizzazione Mondiale della Sanità (OMS), una percentuale significativa di vaccini viene sprecata ogni anno a causa di interruzioni nella catena del freddo.
La criticità principale risiede nella mancanza di visibilità end-to-end, poiché i dati di transito sono spesso disponibili solo a posteriori, impedendo interventi correttivi tempestivi. A ciò si aggiunge un potenziale conflitto di interessi intrinseco alla filiera: il trasportatore, responsabile del mantenimento della temperatura, coincide spesso con la figura deputata a fornire i dati di monitoraggio, creando un incentivo alla manipolazione in caso di guasti o negligenze. Infine, l'eterogeneità dei sistemi informativi (silos) utilizzati da produttori, distributori e farmacie ostacola una comunicazione fluida e in tempo reale, rendendo difficile la ricostruzione precisa della storia termica del prodotto.

\section{Obiettivi del Progetto}
Il sistema proposto mira a risolvere queste problematiche attraverso un approccio integrato che automatizza la fiducia e garantisce l'integrità del dato.

In primo luogo, si punta all'\emph{automazione tramite Smart Contract} per eliminare l'intermediazione umana e burocratica nei processi di verifica e pagamento. Il contratto intelligente agisce come un deposito a garanzia (Escrow), sbloccando i fondi al corriere solo se tutte le condizioni di qualità pattuite vengono matematicamente soddisfatte, rimuovendo così l'arbitrarietà dalla fase di liquidazione.

Parallelamente, viene garantita la \emph{sicurezza del dato} (Data Integrity) assicurando che, una volta acquisita, l'informazione non possa essere alterata. Questo è reso possibile dalla crittografia immodificabile della blockchain e dal meccanismo di consenso IBFT 2.0 di Hyperledger Besu, che rendono il registro distribuito resistente a tentativi di manomissione (Tamper-Proof) ex post.

Infine, l'obiettivo più ambizioso riguarda la \emph{validazione logica} (Data Validity) per garantire che il dato acquisito rifletta fedelmente la realtà. L'Oracolo Bayesiano interviene correlando letture diverse --- come le evidenze di temperatura, umidità, shock, luce e integrità del sigillo --- per calcolare la probabilità posteriore di un evento avverso. Il sistema è quindi in grado di distinguere tra semplici falsi positivi dei sensori e reali compromissioni del carico, basando le proprie decisioni su un'analisi probabilistica robusta delle evidenze raccolte piuttosto che su singole soglie deterministiche.
